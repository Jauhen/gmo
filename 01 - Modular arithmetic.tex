\documentclass[9pt,aspectratio=169]{beamer}

\usepackage{scalerel}

\usetheme{graham}

\title{Modular Arithmetic}
\subtitle[Graham Middle School]{Graham Middle School Math Olympiad Team}

\setcounter{MaxMatrixCols}{20}
\newcommand{\Mod}[1]{\ (\mathrm{mod}\ #1)}
\newcommand{\longdiv}{\smash{\mkern-0.43mu\vstretch{1.31}{\hstretch{.7}{)}}\mkern-5.2mu\vstretch{1.31}{\hstretch{.7}{)}}}}

\begin{document}
\maketitle

\begin{frame}{Modular Arithmetic}
  \begin{columns}[T]
    \begin{column}{0.5\textwidth}
      \begin{problem}
        Now is $4{:}00$ \emph{pm}, what time will be $100$ hours from now? 
      \end{problem}

      \begin{wrapfigure}[6]{r}{0.3\textwidth}\centering
        \vspace*{-1.3ex}
        \includegraphics[width=0.25\textwidth]{01 - Modular arithmetic/clocks.png}
      \end{wrapfigure}

      $4$ days is equal to $4 \times 24 = 96$ hours, so $100$ hours is equal to $4$~days and $4$~hours. That means in $96$~hours we will have the same time. After $4$~hours it will be $8{:}00$~\emph{pm}.\medskip

      \begin{definition}
        \textbf{Modular arithmetic} is a type of arithmetic that deals with remainders of numbers.
      \end{definition}
      
      In modular arithmetic, numbers "wrap around" upon reaching a given number (this given number is known as the \textbf{modulus}) to leave a remainder.

      \begin{definition}
        The value $a \Mod{m}$ is a shorthand way of saying \emph{“the remainder when $a$ is divided by $m$”}, and is from~$0$~to~$m-1$.
      \end{definition}
    \end{column}
    \begin{column}{0.5\textwidth}
      \begin{definition}
        We say that two numbers $a$ and $b$ are \textbf{congruent modulo} $m$ if $b−a$ is
        divisible by $m$. That is,
        \[ a \equiv b \pmod{m} \]
        if and only if $b−a=mk$ for some integer $k$.
      \end{definition}

      \begin{example}
        {\small
        $38$ is congruent to $14 \Mod{12}$, because ${38 − 14 = 24}$, which is a multiple of $12$, or, equivalently, because both $38$ and $14$ have the same remainder $2$ when divided by $12$.  We use a triple equal sign to represent \emph{congruence}: 
        \[38 \equiv 14 \pmod{12}.\]
        The same rule holds for negative values: 
        \begin{align*}
          −8 &\equiv 7\hphantom{-} \pmod{5}; \\
          2 &\equiv −3 \pmod{5}; \\
          −3 &\equiv −8 \pmod{5}.
        \end{align*}
        }
        \vspace*{-2.2ex}
      \end{example}
      In contests, you will often find modular arithmetic a useful tool for solving problems involving remainders or divisibility.
    \end{column}
  \end{columns}
\end{frame}

\begin{frame}{Addition and Multiplication Theorems for Modular Arithmetic}  
  \begin{columns}[T]
    \begin{column}{0.5\textwidth}
      \begin{theorem}
        If $a + b = c$, then 
        \[a\Mod{n} + b\Mod{n} \equiv c\Mod{n}.\]
        To find the \textbf{remainder of the sum} of $a$ and $b$, we can instead \textbf{sum the remainders} of the two terms.  
      \end{theorem}
       
      The proof is straightforward:

      Let's write $a$, $b$ and $c$ as $q_a n + r_a$, $q_b n + r_b$ and $q_a n + r_a$ where $q_a$, $q_b$, and $q_c$ are \emph{quotients} and $r_a$, $r_b$, and $r_c$ are \emph{remainders} of division by $n$.
      Then
      \[ a + b = q_a n + r_a + q_b n + r_b = (q_a + q_b) n + (r_a + r_b) \]
      That means we can ignore $(q_a + q_b) n$ and $q_c n$ when considering numbers \emph{modulo} $n$. So we have 
      \[ r_a + r_b \equiv r_c \pmod{n}. \]
      Which we wanted to proof.
      \begin{example}
        For example:
        \[ 2021 \equiv 2000 + 20 + 1 \pmod{n} \]
        \vspace*{-2.5ex}
      \end{example}
    \end{column}
    \begin{column}{0.5\textwidth}
      \begin{theorem}
        If $a \times b = c$, then 
        \[a\Mod{n} \times b\Mod{n} \equiv c\Mod{n}.\]
        To find the \textbf{remainder of the product} of $a$ and $b$, we can instead \textbf{multiply the remainders} of the two factors.
      \end{theorem}
      Doing the same substitution as in proof of the \emph{addition theorem}
      \begin{multline*}
        a \times b = (q_a n + r_a) (q_b n + r_b) = \\ = q_a q_b n^2 + (q_a r_b + q_b r_a) n + r_a r_b.
      \end{multline*}
      We can ignore $q_a q_b n^2$, $(q_a r_b + q_b r_a) n$, and $q_c n$ because all of them is divisible by $n$. So we have
      \[ r_a \times r_b \equiv r_c \pmod{n}. \]\vspace*{-1ex}
      \begin{example}
        For example:
        \begin{gather*}
          2021 \times 2022 \equiv 21 \times 22 \equiv 462 \equiv 62 \Mod{100}; \\
          42 \times 24 \equiv 3 \times −2 \equiv −6 \equiv 7 \Mod{13}.
        \end{gather*}
        \vspace*{-2.2ex}
      \end{example}
    \end{column}
  \end{columns}
\end{frame}

\begin{frame}{Divisibility rules for powers of 2 and 5}
  \begin{columns}[T]
    \begin{column}{0.5\textwidth}
      \begin{problem}
        Is $721{,}456$ divisible by $8$?
      \end{problem}
      $8$ is $2 \times 2 \times 2$. That means that we can multiply it by $5 \times 5 \times 5$ and get $10 \times 10 \times 10$, so $1,000$ must be divisible by $8$.

      Lets write down $721{,}456$ as $721{,}000 + 456$, as we know from the \emph{addition theorem} the remainder of the division of $721{,}456$ is equal to the sum of remainders of each of summants. Since $21{,}000$ is multiple of $1{,}000$ it remainder modulo $8$ is $0$. So 
      \[ 721{,}456 \equiv 456 \pmod{8}. \]
      That means to check whether $721{,}456$ divisible by $8$, we need to check, whether $456$ is divisible by $8$. Indeed $456 = 8 \times 57$, so $721{,}456$ is also divisible by $8$.\medskip
      
      \begin{definition}
        To know whether a number is divisible by $2^n$, we need to check whether the \textbf{last $n$ digits} of the number is divisible by $2^n$.
      \end{definition}
    \end{column}
    \begin{column}{0.5\textwidth}
      \begin{problem}
        Is $6385$ divisible by $25$?
      \end{problem}
      $25$ is $5 \times 5$, so multiplying it by $2 \times 2$ we will get that $25$ is multiple of $100$.
      Doing the same stuff as with divisibility by $2$, we write $6385$ as $6300 + 85$. $6300$ is a multiple of $100$ thereof it is also multiple of $25$. That means we only need to check whether $85$ is divisible by $25$. $85 = 75 + 10$, so $6385$ is not divisible by $25$.
      \begin{definition}
        To know whether a number is divisible by $5^n$, we need to check whether the \textbf{last $n$ digits} of the number is divisible by $5^n$.
      \end{definition}
      Both rules are true, because $10^n$ is divisible by
      \begin{wrapfigure}{r}{0.42\textwidth}
        \vspace*{-0.8em}
        \hspace*{-1.5em}
        \includegraphics[width=0.55\textwidth]{01 - Modular arithmetic/10-power.png}
      \end{wrapfigure}
      $2^n$~and~$5^n$.~Using~this~facts allows us~to exclude from consideration~all numbers bigger than $10^n$ and that left us with the last $n$ digits of a number.
    \end{column}
  \end{columns}
\end{frame}

\begin{frame}{Divisibility rules for 3 and 9}
  \begin{columns}[T]
    \begin{column}{0.5\textwidth}
      \begin{problem}
        Is $5479$ divisible by $9$?
      \end{problem}
      Let's write a number 5479 as a sum
      \[ 5472 = 5 \cdot 1000 + 4 \cdot 100 + 7 \cdot 10 + 2. \]
      And then write $1000$ as a sum $999 + 1$, $100 = 99 + 1$ and $10 = 9 + 1$. So 
      \begin{multline*}
        5472 = 5 \cdot (999 + 1) + 4 \cdot (99 + 1) + 7 \cdot (9 + 1) + 9 = \\
        = 5 \cdot 999 + 5 + 4 \cdot 99 + 4 + 7 \cdot 9 + 7 + 2.
      \end{multline*}
      And finally we can realize that $999$, $99$, and $9$ are divisible by $9$. That means that divisibility of $5479$ by $9$ depends on divisibility of sum $5 + 4 + 7 + 2$.
      \begin{definition}
        A number is divisible by $9$, if the \textbf{sum of its digits} is divisible by $9$.
      \end{definition}
      And since $9$, $99$, $999$, $9999$ and so on is also divisible by $3$, we get
      \begin{definition}
        A number is divisible by $3$, if the \textbf{sum of its digits} is divisible by $3$.
      \end{definition}
    \end{column}
    \begin{column}{0.5\textwidth}

      \begin{example}
        $\overline{abcd}$ is a mathematical notation that a number is written with digits $a$, $b$, $c$, and $d$. So
        \[ \overline{abcd} = a \cdot 10^3 + b \cdot 10^2 + c \cdot 10 + d. \]
        This allows us to write the rules as
        \[
           \overline{abcd} \equiv a + b + c + d \pmod{9 \text{ or } 3}. 
        \]
        \vspace*{-1em}
      \end{example}\medskip

      The divisibility rule by $9$ allows us quickly check if our arithmetic operations are correct.
      \[
        \begin{array}{c}
        \phantom{\times99}384\\
        \underline{\times\phantom{999}56}\\
        \phantom{\times9}2304\\
        \underline{\phantom\times1820\phantom9}\\
        \phantom\times20504
        \end{array}
      \]
      Lets see remainders of factors and product
      \begin{gather*}
        384 \equiv 6 \Mod{9}, \quad
        56 \equiv 2 \Mod{9}, \\
        20504 \equiv 2 \Mod{9}.
      \end{gather*}
      But $6 \times 2 \not\equiv 2 \Mod{9}$, so there is an error.
    \end{column}
  \end{columns}
\end{frame}

\begin{frame}{Divisibility rules for 11}
  \begin{columns}[T]
    \begin{column}{0.5\textwidth}
      The divisibility rules for $9$ and $3$ works because $10$, $100$, $1{,}000$, and so on are all divisible by $9$ with remainder $1$. But what about $11$?
      \[ 1\underbrace{00\ldots 0}_\text{even zeros} - 1 = \underbrace{99 \ldots 9}_\text{even nines} \]
      is divisible by $11$ because it is sum of
      \[ 99\ 99\ \ldots 99 = 99\cdot 10^n + \ldots + 99 \] which all are divisible by $11$.

      From other hand
      \[ 1\underbrace{00\ldots 0}_\text{odd zeros} + 1 = \underbrace{99\ldots 9}_\text{even nines}0 + 11 \]
      is also divisible by $11$.

      \begin{problem}
        Is $3432$ divisible by $11$?
      \end{problem}
      \vspace*{-1em}
      \begin{multline*}
        3432 = 3 \cdot 1000 + 4 \cdot 100 + 3 \cdot 10 + 2 = \\
        = 3 \cdot (1001 - 1) + 4 \cdot (99 + 1) + 3 \cdot (11 - 1) + 2 \equiv \\
        \text{when we get rid of all multiples of 11} \\
        \equiv -3 + 4 - 3 + 2 = 0 \Mod{11}.
      \end{multline*}
    \end{column}
    \begin{column}{0.5\textwidth}
      As we see our final computation is the digits of the number with alternate signs.
      \begin{definition}
        A number is divisible by $11$ if the \textbf{difference} between the \textbf{sum of the odd-place digits} and the \textbf{sum of the even-place digits} is $0$ or a multiple~of~$11$.
      \end{definition}\medskip

      The combinations of the divisibility rules may also create addition divisibility rules. 

      For examle to check if number is divisible by $6$ we may check if it divisible by $2$ and $3$. This will works when the divisor may be split in product of 
      \begin{wrapfigure}{l}{0.4\textwidth}
        \vspace*{-0.8em}
        \includegraphics[width=0.43\textwidth]{01 - Modular arithmetic/collab.png}
      \end{wrapfigure}
      \textbf{coprime}\footnote{\emph{coprime numbers are numbers that don't have common divisors.}} factors. 
      But the rule will not work for example for $27 = 9 \times 3$, since testing for~$9$ we already testing for $3$.
      \vspace*{1.7em}
    \end{column}
  \end{columns}
\end{frame}

\begin{frame}{Summary of the divisibility rules}
  \begin{columns}[T]
    \begin{column}{0.5\textwidth}
      An integer number $n$ is divisible by
      \begin{enumerate}
        \item[\textbf{2.}] if the \emph{last digit of $n$ is divisble by $2$}, e.g. is equal to $0$, $2$, $4$, $6$, or $8$;
        \item[\textbf{3.}] if the \emph{sum of digits of $n$ is divisible by $3$};
        \item[4.] if the \emph{last two digits of $n$ is divisible by $4$}.\item[\textbf{5.}] if the \emph{last digit of $n$ is divisble by $5$}, e.g. is equal to $0$ or $5$;
        \item[6.] if $n$ is divisible by both $2$ and $3$;
        \item[\textbf{7.}] if \emph{subtracting twice the last digit of $n$ from the remaining digits} gives a multiple of $7$;
        \item[8.] if the \emph{last three digits of $n$ is divisible by $8$};
        \item[9.] if the \emph{sum of digits of $n$ is divisible by $9$};
        \item[10.] if $n$ is divisible by both $2$ and $5$;
        \item[\textbf{11.}] if the \emph{difference of the alternating sum of digits of $n$} is a multiple of $11$;
        \item[12.] if $n$ is divisible by both $3$ and $4$.
      \end{enumerate}
    \end{column}
    \begin{column}{0.5\textwidth}
      {\small
      You might notice the strange divisibility rule for $7$. The proof of this rule is not straightforward as for all other. Let's write $n = a\cdot 10 + b$, where $b$ is the last digit and $a$ is the number build with remaining digits. Then suppose $n$ is divisible by $7$
      \begin{align*}
        a \cdot 10 + b &\equiv 0 \pmod{7}, \text{ multiply all by $2$}\\
        a \cdot 20 + b \cdot 2 &\equiv 0 \pmod{7}, \text{ substract $21a$}\\
        a \cdot (-1) + b \cdot 2 &\equiv 0 \pmod{7}, \text{ multiply by $-1$}\\
        a - b \cdot 2 &\equiv 0 \pmod{7}.
      \end{align*}
      Doing the same operations in reverse order, we can get the divisibility rule for $7$. This is possible because $2$ and $-1$ are \emph{coprime} with $7$. 
      
      The rule may be repeated several times to get a number small enough for the direct check.

      Similar divisibility rules may be obtained for other primes}

      \begin{enumerate}
        \item[\textbf{13.}] if \emph{adding $4$ times the last digit of $n$ to the remaining digits} gives a multiple of $13$;
        \item[\textbf{17.}] if \emph{subtracting $5$ times the last digit of $n$ from the remaining digits} gives a multiple of $17$.
      \end{enumerate}

      {\small But these rules aren't simpler than divide by the number.}
    \end{column}
  \end{columns}
\end{frame}

\begin{frame}{Exercises\hspace*{0.35\textwidth}Challenge Problems}
  \begin{columns}[T]
    \begin{column}{0.5\textwidth}
      \begin{enumerate}
        \item What is the largest integer less than $100$ which is congruent to $3 \Mod{5}$?
        \item How many integers are there between $50$ and $250$ inclusive which are congruent to $1\Mod{7}$?
        \item Find the value of the digit represented by $A$ in the $5$-digit number $12A3B$ if that number is divisible by both $4$ and $9$ and $A$ does not equal $B$.
        \item In how many ways can a debt of $\$69$ be paid exactly using only $\$5$ and $\$2$ bills?
        \item When $n$ is divided by $5$ the remainder is $1$. What is the remainder when $3n$ is divided by~$5$?
        \item The $4$-digit number $4\_89$ is divisible by $11$. What digit goes in the blank?
        \item If the $5$-digit number $3367\_$ is divisible by $15$, what digit must go in the blank?
      \end{enumerate}
    \end{column}
    \begin{column}{0.5\textwidth}
      \begin{enumerate}
        \item The Fibonacci sequence is the sequence $1$, $1$, $2$, $3$, $5$, $8$, $13$, $\dots$, where every term after the second is equal to the sum of the two preceding terms.  Using the addition theorem for modular arithmetic, what is the remainder when the 100\textsuperscript{th} term of the Fibonacci sequence is divided by $8$?  (Hint: \emph{look for a repeating  pattern}).
        \item Prove that the square of any integer is congruent to either $0$, $1$, or $4\Mod{8}$.
      \end{enumerate}
    \end{column}
  \end{columns}
\end{frame}

% \begin{frame}{Title}
%   \begin{columns}[T]
%     \begin{column}{0.5\textwidth}
%     \end{column}
%     \begin{column}{0.5\textwidth}
%     \end{column}
%   \end{columns}
% \end{frame}

\end{document}