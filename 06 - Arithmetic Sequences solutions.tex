\documentclass[9pt,aspectratio=169]{beamer}

\usepackage{scalerel}

\usetheme{graham}

\title{Arithmetic Sequences solutions}
\subtitle[Graham Middle School]{Graham Middle School Math Olympiad Team}

\newcommand\Mydiv[2]{%
$\strut#1$\kern.25em\smash{\raise.3ex\hbox{$\big)$}}$\mkern-8mu
        \overline{\enspace\strut#2}$}

\setcounter{MaxMatrixCols}{20}
\newcommand{\Mod}[1]{\ (\mathrm{mod}\ #1)}
\newcommand{\longdiv}{\smash{\mkern-0.43mu\vstretch{1.31}{\hstretch{.7}{)}}\mkern-5.2mu\vstretch{1.31}{\hstretch{.7}{)}}}}


\DeclareMathOperator{\lcm}{lcm}

\begin{document}
\maketitle

\begin{frame}{Exercises}
  \begin{columns}[T]
    \begin{column}{0.5\textwidth}
      \begin{enumerate}
        \item Find the value of the expression
        
        $100-98+96-94+92-90+\cdots+8-6+4-2.$ %AMC8 2016, Problem 8
        \item The sum of $25$ consecutive even integers is $10,000$. What is the largest of these $25$ consecutive integers? % AMC 8 2016, Problem 19
        \item If the 7th term of an arithmetic sequence is $24$ and the 12th term is $48$, then what is
        the 57th term? % https://www.mathcounts.org/sites/default/files/u50/MCMini3S.pdf
        \item The first two terms of an arithmetic sequence are $4$, $10$. What is the first term greater
        than $1000$? % https://www.mathcounts.org/sites/default/files/u50/MCMini12S.pdf
        \item The sequence an has the property that $a_{n} = a_{n – 1} + 2a_{n-2}$ for $n \geq 2$. It is also
        true that $a_0 = 4$ and $a_4 = 26$. What is the value of $a_{5}$? % Mathcounts 2022, chapter, problem 29
        \seti
      \end{enumerate}
    \end{column}
    \begin{column}{0.5\textwidth}
      \begin{enumerate}
        \conti
        \item The 2020th term of an arithmetic sequence is $\dfrac{21}
        {19}$ times the 2022nd term of the arithmetic sequence.
        What is the value of the 2024th term of this sequence divided by the first term of this sequence? %Mathleague 12111, Problem 10
        \item What is the value of

        $1+2+3-4+5+6+7-8+\cdots+197+198+199-200$? % AMC 10a 2020 Problem 8
        \item What is the greatest number of consecutive integers whose sum is $45$? % AMC 10a 2019 Problem 5
      \end{enumerate}
    \end{column}
  \end{columns}
\end{frame}

\begin{frame}{Challenge problems}
  \begin{columns}[T]
    \begin{column}{0.5\textwidth}
      \begin{enumerate}
        \item In how many ways can $345$ be written as the sum of an increasing sequence of two or more consecutive positive integers? % AMC10b, 2016, Problem 18
        \item $852$ digits are used to number the pages of a book consecutively from page $1$. How many pages
        are there in the book? %AOPS,1, Problem 453.
        \item For every $n$ the sum of $n$ terms of an arithmetic progression is $2n + 3n^2$. What is the $r$th term of
        the sequence in terms of $r$?
        \item A sequence of natural numbers is constructed by listing the first
        $4$, then skipping one, listing the next $5$, skipping $2$, listing $6$, skipping $3$, and on the nth
        iteration, listing $n + 3$ and skipping $n$. The sequence begins $1$, $2$, $3$, $4$, $6$, $7$, $8$, $9$, $10$, $13$.
        What is the $1000$th number in the sequence? % AMC10-2014-A24
      \end{enumerate}
    \end{column}
    \begin{column}{0.5\textwidth}
    \end{column}
  \end{columns}
\end{frame}

\begin{frame}{Exercises 1-3}
  \begin{columns}[T]
    \begin{column}{0.5\textwidth}
      \begin{problem}
        \textbf{E1.} Find the value of the expression
        
        $100-98+96-94+92-90+\cdots+8-6+4-2.$
      \end{problem}
      We can group each subtracting pair together:\[(100-98)+(96-94)+(92-90)+ \ldots +(8-6)+(4-2).\]After subtracting, we have: $2+2+2+\ldots+2+2=2(1+1+1+\ldots+1+1)$. There are $50$ even numbers, therefore there are $25$ pairs. Therefore the sum is $2 \cdot 25=\boxed{50}$

      \begin{problem}
        \textbf{E2.} The sum of $25$ consecutive even integers is $10,000$. What is the largest of these $25$ consecutive integers?
      \end{problem}
      Let $n$ be the 13th consecutive even integer that's being added up. By doing this, we can see that the sum of all 25 $(n-2k)+\dots+(n-4)+(n-2)+(n)+(n+2)+(n+4)+ \dots +(n+2k)=25n$. Now, $25n=10000$, so $n=400$. Remembering that this is the 13th integer, we wish to find the 25th, which is $400+2(25-13)=\boxed{424}$.
    \end{column}
    \begin{column}{0.5\textwidth}
      \begin{problem}
        \textbf{E3.} If the 7th term of an arithmetic sequence is $24$ and the 12th term is $48$, then what is
        the 57th term?
      \end{problem}
      Going from the 7th term to the 12th term of an arithmetic sequence requires adding the
      common difference to the 7th term a total of $12 – 7 = 5$ times. Therefore we know
      $48 = 24 + 5d5 \Rightarrow 24 = 5d \Rightarrow d = 24/5$. To get to the 57th term from the 7th term, the common
      difference of $24/5$ must be added $57 - 7 = 50$ times, resulting in $24 + (24/5)(50) = 24 + 240 = \boxed{264}$.

      
    \end{column}
  \end{columns}
\end{frame}

\begin{frame}{Exercises 3-5}
  \begin{columns}[T]
    \begin{column}{0.5\textwidth}
      \begin{problem}
        \textbf{E4.} The first two terms of an arithmetic sequence are $4$, $10$. What is the first term greater
        than $1000$?
      \end{problem}
      Knowing the first two terms of the arithmetic sequence are $4$, $10$ tells us the common difference is $10 - 4 = 6$. To get from $4$ to $1000$, we will need to add $1000 - 4 = 996$. This means we will have to start with $4$ and add $6$ a total of $996 / 6 = 166$ times. This will get us to exactly
      $1000$. We will first exceed $1000$ by starting with $4$ and adding $6$ a total of $167$ times. The result $is 4 + 6 \cdot 167 = \boxed{1006}$.
    \end{column}
    \begin{column}{0.5\textwidth}
      \begin{problem}
        \textbf{E5.} The sequence an has the property that $a_{n} = a_{n – 1} + 2a_{n-2}$ for $n \geq 2$. It is also
        true that $a_0 = 4$ and $a_4 = 26$. What is the value of $a_{5}$?
      \end{problem}
      \begin{align*}    
        &a_0 = 4;\\
        &a_1 = a_1; \text{ [an as yet unknown value]}\\
        &a_2 = a_1 + 2a_0 = a_1 + 8;\\
        &a_3 = a_2 + 2a_1 = a_1 + 8 + 2a_1 = 3a_1 + 8;\\
        &a_4 = a_3 + 2a_2 = 3a_1 + 8 + 2(a_1 + 8) = 5a_1 + 24 = 26,\\
        &\text{so } a_1 = \frac{26-24}{5} = \frac{2}{5} \text{ and } a_3 = \frac{6}{5} + 8 = \frac{46}{5};\\
        &a_5 = a_4 + 2a_3 = 26 + 2 \times \frac{46}{5} = \frac{130+92}{5} = \boxed{\frac{222}{5}}.
      \end{align*}
    \end{column}
  \end{columns}
\end{frame}

\begin{frame}{Exercises 6-8}
  \begin{columns}[T]
    \begin{column}{0.5\textwidth}
      \begin{problem}
        \textbf{E6.} The 2020th term of an arithmetic sequence is $\dfrac{21}
        {19}$ times the 2022nd term of the arithmetic sequence.
        What is the value of the 2024th term of this sequence divided by the first term of this sequence?
      \end{problem}
      We can suppose that the 2020th term of the arithmetic sequence is $21$ and the 2022nd term of the arithmetic sequence is $19$. The common difference of this arithmetic sequence is $\dfrac{19-21}{2} = -1$. This
      means that the first term of the arithmetic sequence is $21 - 2019d = 2040$ and the 2024th term of
      the arithmetic sequence is $19 + 2d = 17$. The ratio of these terms is $\dfrac{17}{2040} = \boxed{\dfrac{1}{120}}$.
    \end{column}
    \begin{column}{0.5\textwidth}
      \begin{problem}
        \textbf{E7.} What is the value of

        $1+2+3-4+5+6+7-8+\cdots+197+198+199-200$?
      \end{problem}
      $1+2+3-4+5+6+7-8+\cdots+197+198+199-200=$
      $(1 + 2 + 3 + 4 + \cdots + 200) - 2\cdot(4 + 8 + 12 + \cdots + 200) =$ $=\dfrac{(1 + 200)\cdot 200}{2} - 2 \cdot 4 \cdot \dfrac{(1 + 50)\cdot 50}{2} = $ 
      $= 20100 - 10200 = \boxed{9900}.$
      \begin{problem}
        \textbf{E8.} What is the greatest number of consecutive integers whose sum is $45$?
      \end{problem}
      The problem says that they can be integers, not necessarily positive. Observe also that every term in the sequence $-44, -43, \cdots, 44, 45$ cancels out except $45$. Thus, the answer is $\boxed{90}$ integers.
    \end{column}
  \end{columns}
\end{frame}

\begin{frame}{Challenge problems 1-2}
  \begin{columns}[T]
    \begin{column}{0.5\textwidth}
      \begin{problem}
        \textbf{CP1.} In how many ways can $345$ be written as the sum of an increasing sequence of two or more consecutive positive integers?
      \end{problem}
      Factor $345=3\cdot 5\cdot 23$.

      Suppose we take an odd number $k$ of consecutive integers, with the median as $m$. Then $mk=345$ with $\tfrac12k<m$. Looking at the factors of $345$, the possible values of $k$ are $3,5,15,23$ with medians as $115,69,23,15$ respectively.

      Suppose instead we take an even number $2k$ of consecutive integers, with median being the average of $m$ and $m+1$. Then $k(2m+1)=345$ with $k\le m$. Looking again at the factors of $345$, the possible values of $k$ are $1,3,5$ with medians $(172,173),(57,58),(34,35)$ respectively.

      Thus the answer is $\boxed{7}$.
    \end{column}
    \begin{column}{0.5\textwidth}
      \begin{problem}
        \textbf{CP2.} $852$ digits are used to number the pages of a book consecutively from page $1$. How many pages
        are there in the book?
      \end{problem}
      One-digit numbers contribute only $1$ digit, two-digit numbers $2$, and so on. Thus we treat each type separately. The $9$ one-digit numbers each contribute their $1$, for a total $9$ digits; there are $852 - 9 = 843$ to go. The $90$ two-digit numbers from $10$ to $99$ each contribute $2$ digits, for a total of $180$; $843 - 180 = 663$ to go. The $900$ three-digit numbers from $100$ to $999$ contribute $3$ digits each, for $2700$ digits. Thus all the remaining $663$ are three-digit numbers, so $221$ ($=663/3$) are used. We add these $221$ to the last two-digit, $99$, so that the last number used is $99 + 221 = \boxed{320}$.  
    \end{column}
  \end{columns}
\end{frame}

\begin{frame}{Title}
  \begin{columns}[T]
    \begin{column}{0.5\textwidth}
      \begin{problem}
        \textbf{CP3.} For every $n$ the sum of $n$ terms of an arithmetic progression is $2n + 3n^2$. What is the $r$th term of
        the sequence in terms of $r$?
      \end{problem}
      The sum of the first $n$ terms is always $\dfrac{n}{2} (2a + (n-1)d)$, so we have 
      \[ \frac{n}{2}(2a + (n-1)d)=an + dn^2/2 -dn/2. \]
      Thus $d/2 = 3$, so $d = 6$, and $a-d/2 = 2$, so $a=5$. The $r$th term is thus
      \[ a + (r-1)d = 5+6(r-1)=\boxed{6r-1}. \]
    \end{column}
    \begin{column}{0.5\textwidth}
      \begin{problem}
        \textbf{CP4.} A sequence of natural numbers is constructed by listing the first
        $4$, then skipping one, listing the next $5$, skipping $2$, listing $6$, skipping $3$, and on the nth
        iteration, listing $n + 3$ and skipping $n$. The sequence begins $1$, $2$, $3$, $4$, $6$, $7$, $8$, $9$, $10$, $13$.
        What is the $1000$th number in the sequence?
      \end{problem}
      If we list the rows by iterations, then we get

      $1,2,3,4$

      $6,7,8,9,10$

      $13,14,15,16,17,18$ etc.

      so that the $1000$th number is the $16$th number on the $997$th row because $4+5+6+7+\dots+44 = \dfrac{(4+44)\cdot 41}{2}=984$. The last number of the $41$th row (when including the numbers skipped) is $984 + (1+2+3+4+\dots+41)= 1845$, (we add the $1-41$ because of the numbers we skip) so our answer is $1845 + 16 = \boxed{1861}$.
    \end{column}
  \end{columns}
\end{frame}

% \begin{frame}{Title}
%   \begin{columns}[T]
%     \begin{column}{0.5\textwidth}
%     \end{column}
%     \begin{column}{0.5\textwidth}
%     \end{column}
%   \end{columns}
% \end{frame}

\end{document}
