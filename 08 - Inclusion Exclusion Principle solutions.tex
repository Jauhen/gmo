\documentclass[9pt,aspectratio=169]{beamer}

\usepackage{scalerel}

\usetheme{graham}

\title{Inclusion/Exclusion Principle solutions}
\subtitle[Graham Middle School]{Graham Middle School Math Olympiad Team}

\newcommand\Mydiv[2]{%
$\strut#1$\kern.25em\smash{\raise.3ex\hbox{$\big)$}}$\mkern-8mu
        \overline{\enspace\strut#2}$}

\setcounter{MaxMatrixCols}{20}
\newcommand{\Mod}[1]{\ (\mathrm{mod}\ #1)}
\newcommand{\longdiv}{\smash{\mkern-0.43mu\vstretch{1.31}{\hstretch{.7}{)}}\mkern-5.2mu\vstretch{1.31}{\hstretch{.7}{)}}}}


\DeclareMathOperator{\lcm}{lcm}

\begin{document}
\maketitle

% \begin{frame}{Excercises}
%   \begin{columns}[T]
%     \begin{column}{0.5\textwidth}
%       \begin{enumerate}
%         \item Four students take an exam. Three of their scores are $70, 80,$ and $90$. If the average of their four scores is $70$, then what is the remaining score? %AMC 8 2016, Problem 3
%         \item How many subsets of two elements can be removed from the set $\{1, 2, 3, 4, 5, 6, 7, 8, 9, 10, 11\}$ so that the mean (average) of the remaining numbers is $6$? %AMC 8 2015, Problem 13
%         \item Jefferson Middle School has the same number of boys and girls. $\dfrac{3}{4}$ of the girls and $\dfrac{2}{3}$ of the boys went on a field trip. What fraction of the students on the field trip were girls? %AMC8 2016, Problem 12
%         \item The mean, median, and unique mode of the positive integers 3, 4, 5, 6, 6, 7, and $x$ are all equal. What is the value of $x$? %AMC 8 2012, Problem 11
%         \item How many ways to distribute $10$ candies among $4$ kids?
%         \seti
%       \end{enumerate}
%     \end{column}
%     \begin{column}{0.5\textwidth}
%       \begin{enumerate}
%         \conti
%         \item How many ways to distribute $10$ candies among $4$ kids, so each kid should get at least one candy?
%         \item If 20 girls are on my school's soccer team, 25 girls are on my school's hockey team, and 11 girls play both sports, then how many girls play soccer and hockey? % AOPS Interm Counting, Problem 3.1
%         \item At one hospital, there are $100$ patients, all of whom have at least one of the following ailments: a cold, the flu, or an earache. $38$ have a cold, $40$ have the flu, and some number have earaches. If $17$ have both colds and the flu, $10$ have colds and earaches, $23$ have the flu and earaches, and $7$ have all three, how many have an earache? %AOPS Vol 1, Example 27-3
%       \end{enumerate}
%     \end{column}
%   \end{columns}
% \end{frame}

% \begin{frame}{Challenge problems}
%   \begin{columns}[T]
%     \begin{column}{0.5\textwidth}
%       \begin{enumerate}
%         \item One day the Beverage Barn sold 252 cans of soda to 100 customers, and every customer bought at least one can of soda. What is the maximum possible median number of cans of soda bought per customer on that day? %AMC 8 2014, Problem 24
%         \item Mrs. Sanders has three grandchildren, who call her regularly. One calls her every three days, one calls her every four days, and one calls her every five days. All three called her on December 31, 2022. On how many days during the next year did she not receive a phone call from any of her grandchildren? % AMC 8 2017, Problem 24
%         \item A list of $2022$ positive integers has a unique mode, which occurs exactly $10$ times. What is the least number of distinct values that can occur in the list? % AMC 10b 2018, Problem 14
%         \seti
%       \end{enumerate}
%     \end{column}
%     \begin{column}{0.5\textwidth}
%       \begin{enumerate}
%         \conti
%         \item Pat is to select six cookies from a tray containing only
%         chocolate chip, oatmeal, and peanut butter cookies. There are at least six of
%         each of these three kinds of cookies on the tray. How many different assortments
%         of six cookies can be selected? %2003 AMC 10A 21
       

%       \end{enumerate}
%     \end{column}
%   \end{columns}
% \end{frame}

\begin{frame}{Excercises 1-3}
  \begin{columns}[T]
    \begin{column}{0.5\textwidth}
      \begin{problem}
        \textbf{E1.} Four students take an exam. Three of their scores are $70, 80,$ and $90$. If the average of their four scores is $70$, then what is the remaining score?
      \end{problem}
      We can call the remaining score $r$. We also know that the average, $70$, is equal to $\dfrac{70 + 80 + 90 + r}{4}=70$. So $r = \boxed{40}$.
      \begin{problem}
        \textbf{E2.} How many subsets of two elements can be removed from the set $\{1, 2, 3, 4, 5, 6, 7, 8, 9, 10, 11\}$ so that the mean (average) of the remaining numbers is $6$?
      \end{problem}
      Since there will be $9$ elements after removal, and their mean is $6$, we know their sum is $54$. We also know that the sum of the set pre-removal is $66$. Thus, the sum of the $2$ elements removed is $66-54=12$. There are only $\boxed{5}$ subsets of $2$ elements that sum to $12$: $\{1,11\}$, $\{2,10\}$, $\{3, 9\}$, $\{4, 8\}$, $\{5, 7\}$.
    \end{column}
    \begin{column}{0.5\textwidth}
      \begin{problem}
        \textbf{E3.} Jefferson Middle School has the same number of boys and girls. $\dfrac{3}{4}$ of the girls and $\dfrac{2}{3}$ of the boys went on a field trip. What fraction of the students on the field trip were girls?
      \end{problem}
      Let there be $b$ boys and $g$ girls in the school. We see $g=b$, which means $\dfrac{3}{4}b+\dfrac{2}{3}b=\dfrac{17}{12}b$ kids went on the trip and $\dfrac{3}{4}b$ kids are girls. So, the answer is $\dfrac{\dfrac{3}{4}b}{\dfrac{17}{12}b}=\dfrac{9}{17}$, which is $\boxed{\dfrac{9}{17}}$
    \end{column}
  \end{columns}
\end{frame}

\begin{frame}{Exercises 4-6}
  \begin{columns}[T]
    \begin{column}{0.5\textwidth}
      \begin{problem}
        \textbf{E4.} The mean, median, and unique mode of the positive integers 3, 4, 5, 6, 6, 7, and $x$ are all equal. What is the value of $x$?
      \end{problem}
      Notice that the mean of this set of numbers, in terms of $x$, is:
      \[
        \frac{3+4+5+6+6+7+x}{7} = \frac{31+x}{7}
      \]
      Because we know that the mode must be $6$ (it can't be any of the numbers already listed, as shown above, and no matter what $x$ is, either $6$ or a new number, it will not affect $6$ being the mode), and we know that the mode must equal the mean, we can set the expression for the mean and $6$ equal:
      \[
        \frac{31+x}{7} = 6\quad 31+x = 42\quad x = \boxed{11}
      \]
    \end{column}
    \begin{column}{0.5\textwidth}
      \begin{problem}
        \textbf{E5.} How many ways to distribute $10$ candies among $4$ kids?
      \end{problem}
      Using stars and bars we need to distribute $3$ bars across $13$ stars and bars: $\dbinom{13}{3} = \dfrac{13 \cdot 12 \cdot 11}{1 \cdot 2 \cdot 3} = \boxed{286}$.
      \begin{problem}
        \textbf{E6.} How many ways to distribute $10$ candies among $4$ kids, so each kid should get at least one candy?
      \end{problem}
      We just give every kid one candy and distribute the remaining $6$ candies. So $3$ bars and $6$ stars: $\dbinom{9}{3} = \dfrac{9\cdot 8 \cdot 7}{1 \cdot 2 \cdot 3} = \boxed{84}$.
    \end{column}
  \end{columns}
\end{frame}

\begin{frame}{Exercises 7-8}
  \begin{columns}[T]
    \begin{column}{0.5\textwidth}
      \begin{problem}
        \textbf{E7.} If $20$ girls are on my school's soccer team, $25$ girls are on my school's hockey team, and $11$ girls play both sports, then how many girls play soccer and hockey?
      \end{problem}
      $20 - 11 = 9$ girls plays only soccer, $25 - 11 = 14$ plays only hockey. So the total number of girls who play soccer and hockey is $9 + 14 + 11 = \boxed{34}$. Using PIE: $25 + 20 - 11 = \boxed{34}$.
    \end{column}
    \begin{column}{0.5\textwidth}
      \begin{problem}
        \textbf{E8.} At one hospital, there are $100$ patients, all of whom have at least one of the following ailments: a cold, the flu, or an earache. $38$ have a cold, $40$ have the flu, and some number have earaches. If $17$ have both colds and the flu, $10$ have colds and earaches, $23$ have the flu and earaches, and $7$ have all three, how many have an earache?
      \end{problem}
      Let $x$ is the number of patients with earache. Using PIE:
      \begin{gather*}
        x + 38 + 40 - 17 - 10 - 23 + 7 = 100,\\
        x + 35 = 100,\\
        x = \boxed{65}.
      \end{gather*}
    \end{column}
  \end{columns}
\end{frame}

\begin{frame}{Challenge problems 1-2}
  \begin{columns}[T]
    \begin{column}{0.5\textwidth}
      \begin{problem}
        \textbf{CP1.} One day the Beverage Barn sold 252 cans of soda to 100 customers, and every customer bought at least one can of soda. What is the maximum possible median number of cans of soda bought per customer on that day?
      \end{problem}
      In order to maximize the median, we need to make the first half of the numbers as small as possible. Since there are $100$ people, the median will be the average of the $50\text{th}$ and $51\text{st}$ largest amount of cans per person. To minimize the first $49$, they would each have one can. Subtracting these $49$ cans from the $252$ cans gives us $203$ cans left to divide among $51$ people. Taking $\frac{203}{51}$ gives us $3$ and a remainder of $50$. Seeing this, the largest number of cans the $50$th person could have is $3$, which leaves $4$ to the rest of the people. The average of $3$ and $4$ is $3.5$. Thus our answer is $\boxed{3.5}$
    \end{column}
    \begin{column}{0.5\textwidth}
      \begin{problem}
        \textbf{CP2.} Mrs. Sanders has three grandchildren, who call her regularly. One calls her every three days, one calls her every four days, and one calls her every five days. All three called her on December 31, 2022. On how many days during the next year did she not receive a phone call from any of her grandchildren?
      \end{problem}
      We use PIE. The first grandchild called $\left\lfloor\frac{365}{3}\right\rfloor = 121$ days, the second: $\left\lfloor\frac{365}{4}\right\rfloor = 91$, and the third: $\left\lfloor\frac{365}{5}\right\rfloor = 73$. The first and the second called: $\left\lfloor\frac{365}{12}\right\rfloor = 30$, the first and the second called $\left\lfloor\frac{365}{15}\right\rfloor = 24$ and the second and the third called $\left\lfloor\frac{365}{20}\right\rfloor = 18$ days. All three called $\left\lfloor\frac{365}{60}\right\rfloor = 6$ days. As a result she will have $121 + 91 + 73 - 30 - 24 - 18 + 6 = 219$ days with calls and $365 - 219 = \boxed{146}$ days without calls.
    \end{column}
  \end{columns}
\end{frame}

\begin{frame}{Challenge problems 3-4}
  \begin{columns}[T]
    \begin{column}{0.5\textwidth}
      \begin{problem}
        \textbf{CP3.} A list of $2022$ positive integers has a unique mode, which occurs exactly $10$ times. What is the least number of distinct values that can occur in the list?
      \end{problem}
      To minimize the number of distinct values, we want to maximize the number of times they appear. So, we could have $223$ numbers appear $9$ times, $1$ number appear five times, and the mode appear $10$ times, giving us a total of $223 + 1 + 1 = \boxed{225}.$
    \end{column}
    \begin{column}{0.5\textwidth}
      \begin{problem}
        \textbf{CP4.} Pat is to select six cookies from a tray containing only
        chocolate chip, oatmeal, and peanut butter cookies. There are at least six of
        each of these three kinds of cookies on the tray. How many different assortments
        of six cookies can be selected?
      \end{problem}
      Using stars and bars:
      
      Six to stars for the cookies in and two bars to divide the cookies by type. Let the number of chocolate chip cookies be the number of stars to the left of the first divider, the number of oatmeal cookies be the number of stars between the two dividers, and the number of peanut butter cookies be the number of stars to the right of the second divider. There are $\dbinom{8}{2} = \boxed{28}$ ways to place the two dividers, so there are 28 ways to select the six cookies.
    \end{column}
  \end{columns}
\end{frame}

% \begin{frame}{Title}
%   \begin{columns}[T]
%     \begin{column}{0.5\textwidth}
%     \end{column}
%     \begin{column}{0.5\textwidth}
%     \end{column}
%   \end{columns}
% \end{frame}

\end{document}
