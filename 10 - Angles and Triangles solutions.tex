\documentclass[9pt,aspectratio=169]{beamer}

\usepackage{scalerel}

\usetheme{graham}

\title{Angles and Triangles solutions}
\subtitle[Graham Middle School]{Graham Middle School Math Olympiad Team}

\newcommand\Mydiv[2]{%
$\strut#1$\kern.25em\smash{\raise.3ex\hbox{$\big)$}}$\mkern-8mu
        \overline{\enspace\strut#2}$}

\setcounter{MaxMatrixCols}{20}
\newcommand{\Mod}[1]{\ (\mathrm{mod}\ #1)}
\newcommand{\longdiv}{\smash{\mkern-0.43mu\vstretch{1.31}{\hstretch{.7}{)}}\mkern-5.2mu\vstretch{1.31}{\hstretch{.7}{)}}}}

\DeclareMathOperator{\lcm}{lcm}

\usepackage{luamplib}
\mplibsetformat{metafun}
\mplibtextextlabel{enable}
\everymplib{input repere; input macros; beginfig(1);}
\everyendmplib{endfig;}

\begin{document}
\maketitle

\begin{frame}{Problems 1-3}
  \begin{columns}[T]
    \begin{column}{0.5\textwidth}
      \begin{problem}
        \textbf{E1.} If the degree measures of the angles of a triangle are in the ratio $3:3:4$, what is the degree measure of the largest angle of the triangle?
      \end{problem}\pause
      Let the angles are $3x$, $3x$, and $4x$, so $3x + 3x + 4x = 10x = 180°$. So $x = 18°$ and $4x = \boxed{72°}$.\pause

      \begin{problem}
        \vspace*{1ex}
        \begin{columns}[T, totalwidth=0.95\textwidth]
          \hspace{0.7ex}
          \begin{column}{0.6\linewidth}
            \textbf{E2.} In $\bigtriangleup ABC$, $D$ is a point on side $\overline{AC}$ such that $BD=DC$ and $\angle BCD$ measures $70°$. What is the degree measure of $\angle ADB$?
          \end{column}
          \hspace{0.7ex}
          \begin{column}{0.35\linewidth}
            \leavevmode
            \begin{mplibcode}
              u = 0.33cm;
              pair A, B, C, D;
              A := origin;
              C := (7u, 0u);
              D := (3u, 0u);
              B := whatever[D, D + u*dir(40)] = whatever[C, C + u*dir(110)];
              Draw A--B--C--cycle, B--D;
              Dot A, B, C, D;
              rimmark(B--D, D--C);
              labelarcs(A, C, B, 6, "$70°$");
              label.bot("$A$", A);
              label.bot("$D$", D);
              label.bot("$C$", C);
              label.top("$B$", B);
            \end{mplibcode} %AMC 8 2014 Problem 9
          \end{column}
        \end{columns}
      \end{problem}\pause
      $\angle ADB = \angle DCB + \angle DBC$, and since $\triangle BDC$ is isosceles, $\angle DBC = \angle DCB = 70°$, so $\angle ADB = 70° + 70° = \boxed{140°}$.\pause
    \end{column}
    \begin{column}{0.5\textwidth}
      \begin{problem}
        \textbf{E3.} How many scalene triangles have all sides of integral lengths and perimeter less than $15$?
      \end{problem}\pause
      Let's look at the longest side of a triangle. Then iterate on the length of 
      the middle side. 

      Case 0: $3$ or smaller. The only option $3-2-1$ is degenerated triangle.

      Case 1: $4$, the only one scalene triangle possible: $4-3-2$.

      Case 2: $5$: $5-4-2$ and $5-4-3$.

      Case 3: $6$: $6-4-3$, $6-5-2$, and $6-5-3$.

      Case 4: $7$: the sum two other sides will be at least $8$.

      As a result we have $\boxed{6}$ scalene triangles.
    \end{column}
  \end{columns}
\end{frame}

\begin{frame}{Problems 4-6}
  \begin{columns}[T]
    \begin{column}{0.5\textwidth}
      \begin{problem}
        \textbf{E4.} On sides $AB$ and $AC$ of $\triangle ABC$, we pick points $D$ and $E$, respectively, so that
        $DE \parallel BC$. If $AB = 3AD$ and $DE = 6$, find $BC$.
      \end{problem}\pause

      \begin{wrapfigure}{l}{10.00mm}
        \vspace*{-\intextsep}
        \begin{mplibcode}
          u := 0.5cm;
          pair A, B, C, D, E;
          A := origin;
          B := 3u*dir(65);
          C := 4u*dir(0);
          D := 1/3[A, B];
          E := 1/3[A, C];
          Draw A--B--C--cycle, D--E;
          Dot A, B, C, D, E;
          label.bot("$A$", A);
          label.ulft("$D$", D);
          label.ulft("$B$", B);
          label.bot("$C$", C);
          label.bot("$E$", E);
        \end{mplibcode}
      \end{wrapfigure}\pause
      Since $\triangle ADE \sim \triangle ABC$ via \textbf{AA} similarity, $DE : BC = AD : AB$, so $BC = DE \times AB / AD = 6 \times 3 = \boxed{18}$.\pause
      \begin{problem}
        \textbf{E5.} Let $\angle ABC = 24°$ and $\angle ABD = 20°$. What is the smallest possible degree measure for $\angle CBD$?
      \end{problem}\pause

      $\angle ABD$ and $\angle ABC$ share ray $AB$. In order to minimize the value of $\angle CBD$, $D$ should be located between $A$ and $C$.$\angle ABC = \angle ABD + \angle CBD$, so $\angle CBD = 4$. The answer is $\boxed{4}$.\pause

    \end{column}
    \begin{column}{0.5\textwidth}
      \begin{problem}
        \textbf{E6.} A polygon has $N$ sides and $q$ obtuse interior angles. Each of its obtuse interior angles has a measure $150°$ and each of its acute interior angles has measure $80°$. How many sides does the polygon have?
      \end{problem}\pause
      The total sum of angles in the polygon is $180 \cdot (N - 2)$ and from another side, the sum of angles is $150 \cdot q + 80 \cdot (N - q)$.
      \begin{gather*}
        180 (N - 2) = 150 q + 80 (N - q),\\
        180 N - 360 = 150 q + 80 N - 80 q,\\
        100 N - 360 = 70 q, \\
        100 N = 360 + 70 q, \\
        10 N = 36 + 7q.
      \end{gather*}\pause
      So $7q$ should be ended up with $4$. This is possible when $q = 2, 12, 22$ etc. For $q = 2$, $\boxed{N = 5}$, for $q = 12$, $\boxed{N = 12}$, but for each $q> 12$, $N < q$, so the there are no more solutions. 
    \end{column}
  \end{columns}
\end{frame}

\begin{frame}{Problems 7-8}
  \begin{columns}[T]
    \begin{column}{0.5\textwidth}
      \begin{problem}
        \textbf{E7.} In how many ways can we form a nondegenerate triangle by choosing three distinct
        numbers from the set $\{1,\ 2,\ 3,\ 4,\ 5\}$ as the sides?
      \end{problem}\pause
      Let's look at the longest side: 

      Case 1: The longest side is $5$. The sum of two other sides should be at least $6$, so the only options are $2-4-5$ and $3-4-5$.

      Case 2: The longest side is $4$. The sum of two other sides should be at least $5$, so the only option is $2-3-4$.

      Case 3: The longest side is $3$. The rest two sides are $1$ and $2$ and we can got only degenerated triangle.

      So we have only three cases $2-4-5$, $3-4-5$, and $2-3-4$.\pause
    \end{column}
    \begin{column}{0.5\textwidth}
      \begin{problem}
        \textbf{E8.} The ratio of the measures of two acute angles is $5:4$, and the complement of one of these two angles is twice as large as the complement of the other. What is the sum of the degree measures of the two angles?
      \end{problem}\pause
      We let the measures be $5x$ and $4x$ giving us the ratio of $5:4$. We know $90-4x>90-5x$ since this inequality gives $x>0$, which is true since the measures of angles are never negative. We also know the bigger complement is twice the smaller, so
      \begin{gather*}
        90-4x=2(90-5x),\\
        90-4x=180-10x,\\
        6x=90,\\
        x=15.
      \end{gather*}
      Therefore, the angles are $75$ and $60$, which sum to $\boxed{135}$.
    \end{column}
  \end{columns}
\end{frame}

\begin{frame}{Challenge problems 1-2}
  \begin{columns}[T]
    \begin{column}{0.5\textwidth}
      \begin{problem}
        \textbf{C1.} Joy has $30$ thin rods, one each of every integer length from $1$ cm through $30$ cm. She places the rods with lengths $3$ cm, $7$ cm, and $15$ cm on a table. She then wants to choose a fourth rod that she can put with these three to form a quadrilateral with positive area. How many of the remaining rods can she choose as the fourth rod?
      \end{problem}\pause
      The triangle inequality generalizes to all polygons, so $x < 3+7+15$ and $15<x+3+7$ yields $5<x<25$. Now, we know that there are $19$ numbers between $5$ and $25$ exclusive, but we must subtract $2$ to account for the 2 lengths already used that are between those numbers, which gives $19-2=\boxed{17}$.\pause
    \end{column}
    \begin{column}{0.5\textwidth}
      \begin{problem}
        \textbf{C2.} Right triangle $ABC$ has leg lengths $AB=20$ and $BC=21$. Including $\overline{AB}$ and $\overline{BC}$, how many line segments with integer length can be drawn from vertex $B$ to a point on hypotenuse $\overline{AC}$? 
      \end{problem}\pause
      $AC = \sqrt{20^2 + 21^2} = 29$ and $\triangle ABC \sim \triangle APB$, so $BP = BC \times AB / AC = 21 \times 20 / 29 \approx 14.5$.\pause
      \begin{wrapfigure}{l}{20.00mm}
        \vspace*{-\intextsep}
        \begin{mplibcode}
          u = 0.1cm;
          pair A, B, C, E, P; 
          A:=(-20u, 0u); 
          B:=origin; 
          C:=(0u,21u); 
          E:=(-21u, 20u); 
          P:=altitude(A, B, C); 
          draw A--B--C--cycle; 
          draw B--P; 
          label.llft("$A$", A); 
          label.lrt("$B$", B); 
          label.urt("$C$", C); 
          label.bot("$P$", P);
          draw subpath (1, 2) of circle(B, 21u);
          draw subpath (1, 2) of circle(B, 20u);
          draw subpath (1, 2) of circle(B, 19u);
          draw subpath (1, 2) of circle(B, 18u);
          draw subpath (1, 2) of circle(B, 17u);
          draw subpath (1, 2) of circle(B, 16u);
          draw subpath (1, 2) of circle(B, 15u);
        \end{mplibcode}          
      \end{wrapfigure}\pause
      It follows that we can draw circles of radii $15, 16, 17, 18, 19,$ and $20,$ that each contribute two integer lengths (since these circles intersect the hypotenuse twice) from $B$ to $\overline{AC}$ and one circle of radius $21$ that contributes only one such segment. Our answer is then $6 \cdot 2 + 1 = \boxed{13}$.
    \end{column}
  \end{columns}
\end{frame}

\begin{frame}{Challenge problem 3}
  \begin{columns}[T]
    \begin{column}{0.5\textwidth}
      \begin{problem}
        \textbf{C3.} A square with side length $x$ is inscribed in a right triangle with sides of length $3$, $4$, and $5$ so that one vertex of the square coincides with the right-angle vertex of the triangle. A square with side length $y$ is inscribed in another right triangle with sides of length $3$, $4$, and $5$ so that one side of the square lies on the hypotenuse of the triangle. What is $\dfrac{x}{y}$?
      \end{problem}\pause
      \begin{columns}
        \begin{column}[T]{0.5\textwidth}          
          \begin{mplibcode}
            u = 0.6cm;
            pair A,B,C; pair D, e, F; 
            A := (0,0); 
            B := (4u,0); 
            C := (0,3u);  
            D := (0, 12/7u); 
            e := (12/7u, 12/7u); 
            F := (12/7u, 0);  
            draw A--B--C--cycle; 
            draw D--e--F;  
            label.lft("$x$", D/2); 
            label.llft("$A$", A); 
            label.bot("$B$", B); 
            label.top("$C$", C); 
            label.lft("$D$", D); 
            label.urt("$E$", e); 
            label.bot("$F$", F);
          \end{mplibcode}\pause
        \end{column}
        \begin{column}[T]{0.5\textwidth}
          \begin{mplibcode}
            u = 0.6cm;
            pair A,B,C; pair q, R, S, T; 
            A := (0,0); 
            B := (4u,0); 
            C := (0,3u);  
            q := (1.297u, 0); 
            R := (2.27u, 1.297u); 
            S := (0.973u, 2.27u); 
            T := (0, 0.973u); 
            draw A--B--C--cycle; 
            draw q--R--S--T--cycle;  
            label.ulft("$y$", (q+R)/2); 
            label.llft("$A'$", A); 
            label.lrt("$B'$", B); 
            label.top("$C'$", C); 
            label.("$Q$", (q-(0,0.3u))); 
            label.urt("$R$", R); 
            label.urt("$S$", S); 
            label.lft("$T$", T);
          \end{mplibcode}
        \end{column}
      \end{columns}
    \end{column}
    \begin{column}{0.5\textwidth}
      Analyze the first right triangle.
      Note that $\triangle ABC$ and $\triangle FBE$ are similar, so $\frac{BF}{FE} = \dfrac{AB}{AC}$. This can be written as $\dfrac{4-x}{x}=\dfrac{4}{3}$. Solving, $x = \dfrac{12}{7}$.\pause
      
      Now we analyze the second triangle.

      Similarly, $\triangle A'B'C'$ and $\triangle RB'Q$ are similar, so $RB' = \dfrac{4}{3}y$, and $C'S = \dfrac{3}{4}y$. Thus, $C'B' = C'S + SR + RB' = \dfrac{4}{3}y + y + \dfrac{3}{4}y = 5$. Solving for $y$, we get $y = \dfrac{60}{37}$. Thus, $\dfrac{x}{y} = \boxed{\frac{37}{35}}$.
    \end{column}
  \end{columns}
\end{frame}

\begin{frame}{Challenge problem 4}
  \begin{columns}[T]
    \begin{column}{0.5\textwidth}
      \begin{problem}
        \textbf{C4.} Quadrilateral $ABCD$ has $AB = BC = CD$, angle $ABC = 70°$ and angle $BCD = 170°$. What is the measure of angle $BAD$?
      \end{problem}\pause
      \begin{wrapfigure}{l}{20.00mm}
        \begin{mplibcode}
          u = 0.5cm; 
          a:=4u; 
          pair A, B, C, D, E;
          A:=(0,0); 
          B = a*dir(0);
          C=B+a*dir(110); 
          D=C+a*dir(120); 
          E=D+a*dir(0); 
          Draw A--B--C--D--cycle, E--C, B--D, B--E, D--E; 
          label.llft("$A$",A); 
          label.lrt("$B$",B); 
          label.lrt("$C$",C); 
          label.top("$D$",D); 
          label.urt("$E$",E); 
          label("$60°$",C + .75u*dir(360-65-115-55-30)); 
          label("$65°$",B + .75u*dir(180-32.5)); 
          label("$x°$",A + .5u*dir(42.5)); 
          label("$5°$",D + 2.5u*dir(360-60-2.5)); 
          label("$60°$",D + .75u*dir(360-30)); 
          label("$60°$",E + .5u*dir(360-150)); 
          label("$5°$",B + 2.5u*dir(180-65-2.5));
        \end{mplibcode}
        \vspace*{-\intextsep}
      \end{wrapfigure}\pause
      First, connect the diagonal $DB$, then, draw line $DE$ such that it is congruent to $DC$ and is parallel to $AB$. Because triangle $DCB$ is isosceles and angle $DCB$ is $170°$, the angles $CDB$ and $CBD$ are both $\dfrac{180-170}{2} = 5°$. Because angle $ABC$ is $70°$, we get angle $ABD$ is $65°$. Next, noticing parallel lines $AB$ and $DE$ and transversal $DB$, we see that angle $BDE$ is also $65°$, and subtracting off angle $CDB$ gives that angle $EDC$ is $60°$.
    \end{column}
    \begin{column}{0.5\textwidth}
  
Now, because we drew $ED = DC$, triangle $DEC$ is equilateral. We can also conclude that $EC=DC=CB$ meaning that triangle $ECB$ is isosceles, and angles $CBE$ and $CEB$ are equal.

Finally, we can set up our equation. Denote angle $BAD$ as $x°$. Then, because $ABED$ is a parallelogram, the angle $DEB$ is also $x°$. Then, $CEB$ is $(x-60)°$. Again because $ABED$ is a parallelogram, angle $ABE$ is $(180-x)°$. Subtracting angle $ABC$ gives that angle $CBE$ equals $(110-x)°$. Because angle $CBE$ equals angle $CEB$, we get\[x-60=110-x,\] solving into $x=\boxed{85°}$.
    \end{column}
  \end{columns}
\end{frame}

\begin{frame}{Title}
  \begin{columns}[T]
    \begin{column}{0.5\textwidth}
    \end{column}
    \begin{column}{0.5\textwidth}
    \end{column}
  \end{columns}
\end{frame}

\begin{frame}{Team attack 4 solutions, problems 1-2}
  \begin{columns}[T]
    \begin{column}{0.5\textwidth}
      \begin{problem}
        \textbf{TA1.} It takes Mary $30$ minutes to walk uphill $1$ km from her home to school, but it takes her only $10$ minutes to walk from school to her home along the same route. What is her average speed, in km/hr, for the round trip?
      \end{problem}\pause
      Since she walked $1$ km to school and $1$ km back home, her total distance is $1+1=2$ km.

      Since she spent $30$ minutes walking to school and $10$ minutes walking back home, her total time is $30+10=40$ minutes = $\dfrac{40}{60}=\dfrac{2}{3}$ hours.

      Therefore her average speed in km/hr is $\dfrac{2}{\frac{2}{3}}=\boxed{3}$.\pause
    \end{column}
    \begin{column}{0.5\textwidth}
      \begin{problem}
        \textbf{TA2.} Members of the Rockham Soccer League buy socks and T-shirts. Socks cost \$4 per pair and each T-shirt costs \$5 more than a pair of socks. Each member needs one pair of socks and a shirt for home games and another pair of socks and a shirt for away games. If the total cost is \$2366, how many members are in the League?
      \end{problem}\pause
      Since T-shirts cost $5$ dollars more than a pair of socks, T-shirts cost $5+4=9$ dollars.

      Since each member needs $2$ pairs of socks and $2$ T-shirts, the total cost for $1$ member is $2(4+9)=26$ dollars.

      Since $2366$ dollars was the cost for the club, and $26$ was the cost per member, the number of members in the League is $2366\div 26=\boxed{91}$.
    \end{column}
  \end{columns}
\end{frame}

\begin{frame}{Team attack 4 solutions, problems 3-4}
  \begin{columns}[T]
    \begin{column}{0.5\textwidth}
      \begin{problem}
        \textbf{TA3.} How many non-congruent triangles with perimeter $7$ have integer side lengths?
      \end{problem}\pause
      By the triangle inequality, no side may have a length greater than the semiperimeter, which is $\dfrac{1}{2}\cdot7=3{.}5$.

      Since all sides must be integers, the largest possible length of a side is $3$. Therefore, all such triangles must have all sides of length $1$, $2$, or $3$. Since $2+2+2=6<7$, at least one side must have a length of $3$. Thus, the remaining two sides have a combined length of $7-3=4$. So, the remaining sides must be either $3$ and $1$ or $2$ and $2$. Therefore, the number of triangles is $\boxed{2}$.\pause
    \end{column}
    \begin{column}{0.5\textwidth}
      \begin{problem}
        \textbf{TA4.} What is the probability that a randomly drawn positive factor of $60$ is less than $7$?
      \end{problem}\pause
      Notice that $1 \cdot 60=2 \cdot 30=3 \cdot 20=4 \cdot 15=5 \cdot 12=6 \cdot 10=60$. Hence, $60$ has $12$ factors, of which $6$ are less than $7$. Thus, the answer is $\dfrac{6}{12} = \boxed{\dfrac{1}{2}}$.\pause

      \begin{problem}
        \textbf{TA6.} The sum of the two 5-digit numbers $AMC10$ and $AMC12$ is $123422$. What is $A+M+C$?
      \end{problem}\pause
      We know that $AMC12$ is $2$ more than $AMC10$. We set up $AMC10=x$ and $AMC12=x+2$. We have $x+x+2=123422$. Solving for $x$, we get $x=61710$. Therefore, the sum $A+M+C= \boxed{14}$.
    \end{column}
  \end{columns}
\end{frame}


\begin{frame}{Team attack 4 solutions, problems 5-6}
  \begin{columns}[T]
    \begin{column}{0.5\textwidth}
      \begin{problem}
        \textbf{TA5.} The polygon enclosed by the solid lines in the figure consists of 4 congruent squares joined edge-to-edge. One more congruent square is attached to an edge at one of the nine positions indicated. How many of the nine resulting polygons can be folded to form a cube with one face missing?
      \end{problem}\pause
      \begin{center}
        \leavevmode
        \begin{mplibcode}
          u = 0.6cm;
          fill (1u,1u)--(2u,1u)--(2u,2u)--(4u,2u)--(4u,3u)--(1u,3u)--cycle withcolor 0.5white; 
          draw (1u,1u)--(2u,1u)--(2u,2u)--(4u,2u)--(4u,3u)--(1u,3u)--cycle;
          draw (0u,1u)--(0u,3u)--(1u,3u)--(1u,4u)--(4u,4u)--(4u,3u)--(5u,3u)--(5u,2u)--(4u,2u)--(4u,1u)--(2u,1u)--(2u,0u)--(1u,0u)--(1u,1u)--cycle dashed evenly; 
          draw (0u,2u)--(2u,2u)--(2u,4u) dashed evenly; 
          draw (3u,1u)--(3u,4u) dashed evenly; 
          label("$1$",(1.5u,0.5u)); 
          draw circle((1.5u,0.5u),.3u); 
          label("$2$",(2.5u,1.5u)); 
          draw circle((2.5u,1.5u),.3u); 
          label("$3$",(3.5u,1.5u)); 
          draw circle((3.5u,1.5u),.3u); 
          label("$4$",(4.5u,2.5u)); 
          draw circle((4.5u,2.5u),.3u); 
          label("$5$",(3.5u,3.5u)); 
          draw circle((3.5u,3.5u),.3u); 
          label("$6$",(2.5u,3.5u)); 
          draw circle((2.5u,3.5u),.3u); 
          label("$7$",(1.5u,3.5u)); 
          draw circle((1.5u,3.5u),.3u); 
          label("$8$",(0.5u,2.5u)); 
          draw circle((0.5u,2.5u),.3u); 
          label("$9$",(0.5u,1.5u)); 
          draw circle((0.5u,1.5u),.3u);
        \end{mplibcode}
        \vspace*{-0.5ex}
      \end{center}
      A cube missing one face has $5$ of its $6$ faces. Since the shape has $4$ faces already, we need another face. The only way to add another face is if the added square does not overlap any of the others. $1$, $2$, and $3$ overlap, while squares $4$ to $9$ do not. The answer is $\boxed{6}$.\pause
    \end{column}
    \begin{column}{0.5\textwidth}
      \begin{problem}
        \textbf{TA7.} A point $(x,\ y)$ is randomly picked from inside the rectangle with vertices $(0,\ 0)$, $(4,\ 0)$, $(4,\ 1)$, and $(0,\ 1)$. What is the probability that $x<y$?
      \end{problem}\pause
      \begin{center}
        \leavevmode
        \begin{mplibcode}
          u = 1cm;
          fill (0u, 0u)--(4u, 0u)--(4u, 1u)--(1u, 1u)--cycle withcolor 0.6white;
          draw  (0u, 0u)--(4u, 0u)--(4u, 1u)--(0u, 1u)--cycle;
          drawarrow (-1u, 0u)--(5u, 0u);
          drawarrow (0u, -0.5u)--(0u, 1.7u);
          label.llft("$x$", (5u, 0u));
          label.llft("$y$", (0u, 1.7u));
          label.bot("$4$", (4u, 0u));
          label.lft("$1$", (0u, 1u));
          label.llft("$0$", (0u, 0u));
        \end{mplibcode}
      \end{center}
      The area of this rectangle is $4\cdot1=4$.

      The line $x=y$ intersects the rectangle at $(0,0)$ and $(1,1)$.

      The area of this triangle is $\dfrac{1}{2}\cdot1^{2}=\dfrac{1}{2}$

      Therefore, the probability that $x<y$ is $\dfrac{\frac{1}{2}}{4}=\boxed{\dfrac{1}{8}}$.
    \end{column}
  \end{columns}
\end{frame}

% \begin{frame}{Title}
%   \begin{columns}[T]
%     \begin{column}{0.5\textwidth}
%     \end{column}
%     \begin{column}{0.5\textwidth}
%     \end{column}
%   \end{columns}
% \end{frame}



\end{document}