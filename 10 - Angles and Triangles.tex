\documentclass[9pt,aspectratio=169]{beamer}

\usepackage{tabularx}
\newcolumntype{Y}{>{\centering\arraybackslash\leavevmode}X}
\usepackage{luamplib}
\everymplib{input mpcolornames; input repere; beginfig(1);}
\everyendmplib{endfig;}

\usetheme{graham}

\title{Angles and Triangles}
\subtitle[Graham Middle School]{Graham Middle School Math Olympiad Team}

\begin{document}
\maketitle

\begin{frame}{Types of angles and their measures}
  \begin{columns}[T]
    \begin{column}{0.5\textwidth}
      We typically measure angles in degrees (symbol~$°$), with an entire circle having a measure of $360°$.  A pair of rays intersecting to form a~straight line therefore form a 180° angle. The number $360$ is somewhat arbitrary.  It was developed in ancient Babylonia where they used a sexigesimal (base $60$) number system and had a $360$ day calendar.

      \begin{example}
        Angles are sometimes also measured in radians, where $2\pi$ radians is equal to $360$ degrees.
      \end{example}
    \end{column}
    \begin{column}{0.5\textwidth}
      Angles are classified as acute, right, or obtuse depending on whether they measure less than, equal to, or greater than $90°$ respectively.\medskip

      \begin{tabularx}{\textwidth}{YYY}
        \begin{mplibcode}
          u = 0.6cm;
          repere(-1,5,u,-1,5,u);
            drawarrow (0, 0)--(2 * dir(40)) withpen pencircle scaled 1.25;
            drawarrow (0, 0)--(2, 0) withpen pencircle scaled 1.25;
            pickup pencircle scaled 2.5;
            drawdot (0, 0);
          fin;
        \end{mplibcode}
        &
        \begin{mplibcode}
          u = 0.6cm;
          repere(-1,5,u,-1,5,u);
            drawarrow (0, 0)--(2 * dir(90)) withpen pencircle scaled 1.25;
            drawarrow (0, 0)--(2, 0) withpen pencircle scaled 1.25;
            draw marqueangledroit((2, 0), (0, 0), (0, 2));
            pickup pencircle scaled 2.5;
            drawdot (0, 0);
          fin;
        \end{mplibcode}
        &
        \begin{mplibcode}
          u = 0.6cm;
          repere(-1,5,u,-1,5,u);
            drawarrow (0, 0)--(2 * dir(120)) withpen pencircle scaled 1.25;
            drawarrow (0, 0)--(2, 0) withpen pencircle scaled 1.25;
            pickup pencircle scaled 2.5;
            drawdot (0, 0);
          fin;
        \end{mplibcode} \\
        Acute & Right & Obtuse
      \end{tabularx}
      \begin{definition}
        Pairs of angles whose measures sum to $90°$ are called \textbf{complementary} angles.  
        
        Pairs of angles that sum to $180°$ are called \textbf{supplementary}.
      \end{definition}
    \end{column}
  \end{columns}
\end{frame}

\begin{frame}{Vertical angles and angles formed by parallel lines}
  \begin{columns}[T]
    \begin{column}{0.5\textwidth}
      \begin{center}
        \leavevmode
        \begin{mplibcode}
          u = 1cm;
          repere(-5,5,u,-5,5,u);
            pair O, A, B, C, D;
            O := (0, 0);
            A := (2 * dir(25));
            B := (2 * dir(-25));
            C := (-2 * dir(25));
            D := (-2 * dir(-25));
            drawarrow O--A withpen pencircle scaled 1.25;
            drawarrow O--B withpen pencircle scaled 1.25;
            drawarrow O--C withpen pencircle scaled 1.25;
            drawarrow O--D withpen pencircle scaled 1.25;
            pickup pencircle scaled 2.5;
            drawdot O;
            nomme(B, O, A, btex $\phi$ etex);
            nomme(D, O, C, btex $\phi'$ etex);
            label.top(btex $\theta$ etex, O + 0.15 * dir(90));
            label.bot(btex $\theta'$ etex, O + 0.15 * dir(-90));
          fin;
        \end{mplibcode}
      \end{center}

      Two pairs of vertical angles ($\theta$ and $\theta'$, $\phi$ and $\phi'$) are formed by intersecting lines.  Since they are angles that make a line, $\theta$ and $\phi$ sum to $180°$.  Likewise, $\theta'$ and $\phi$ sum to $180°$.  Therefore $\theta = \theta'$ and the angles are said to be congruent.  Since $\phi$ and $\phi'$ both form lines when combined with $\theta$, we also see that $\phi = \phi'$.

      \begin{definition}
        Intersecting lines form two pairs of congruent \textbf{vertical} angles.
      \end{definition}

    \end{column}
    \begin{column}{0.5\textwidth}
      In the figure below, $l$ and $m$ are parallel lines and line $n$ is called a \textbf{transversal}.  
      \begin{center}
        \leavevmode
        \begin{mplibcode}
          u = 0.4cm;
          repere(-5,5,u,-5,5,u);
            path l, m, n;
            l := (-5, 1.5)--(5, 1.5);
            m := (-5, -1.5)--(5, -1.5);
            n := (2, 4)--(-2, -4);
            draw l withpen pencircle scaled 1.25;
            draw m withpen pencircle scaled 1.25;
            draw n withpen pencircle scaled 1.25;
            label.top(btex $l$ etex, point 0.05 of l);
            label.top(btex $m$ etex, point 0.05 of m);
            label.lrt(btex $n$ etex, point 0.05 of n);
            pair A, B;
            A := l intersectionpoint n;
            B := m intersectionpoint n;
            label.lrt(btex $\theta$ etex, A);
            label.llft(btex $\beta$ etex, A + 0.5 * dir(180));
            label.llft(btex $\gamma$ etex, B + 0.3 * dir(180));
            label.urt(btex $\alpha$ etex, B + 0.5 * dir(0));
          fin;
        \end{mplibcode}
      \end{center}
      
      \begin{definition}
        Angles $\alpha$ and $\beta$ are called \textbf{alternate interior angles} and they are congruent. 
        
        Angles in same relative locations to lines $l$ and $m$ respectively are \textbf{corresponding angles}, such as $\gamma$ and $\beta$, and they are also congruent.  
      \end{definition}
    \end{column}
  \end{columns}
\end{frame}

\begin{frame}{Types of triangles}
  \begin{columns}[T]
    \begin{column}{0.5\textwidth}
      Any $3$ points that \emph{don’t lie on the same line} can be vertices of a triangle.  
      
      The length of any side in a triangle must be \textbf{less than the sum of the lengths of the other two sides}.
      \begin{definition}
        Triangles can be classified by the number of sides of equal length.
      \end{definition}

      \begin{tabularx}{\textwidth}{YYY}
        \begin{mplibcode}
          u = 0.5cm;
          taille_marque_s := 0.15cm;
          angle_marque_s := 90;
          repere(-1,5,u,-1,5,u);
            pair A, B, C;
            A := origin;
            B := 4 * dir(60);
            C := 4 * dir(0);
            draw triangle(A, B, C) withpen pencircle scaled 1.25;
            nomme.top(A, btex $$ etex);
            nomme.top(B, btex $$ etex);
            nomme.top(C, btex $$ etex);
            draw marquesegment(A, B, 1);
            draw marquesegment(A, C, 1);
            draw marquesegment(C, B, 1);
            draw marqueangle(A, B, C, 1);
            draw marqueangle(B, C, A, 1);
            draw marqueangle(C, A, B, 1);
          fin;
        \end{mplibcode}
        &
        \begin{mplibcode}
          u = 0.5cm;
          taille_marque_s := 0.15cm;
          angle_marque_s := 90;
          repere(-1,5,u,-1,5,u);
            pair A, B, C;
            A := origin;
            B := 4.2 * dir(65);
            C := B + 4.2 * dir(-65);
            draw triangle(A, B, C) withpen pencircle scaled 1.25;
            nomme.top(A, btex $$ etex);
            nomme.top(B, btex $$ etex);
            nomme.top(C, btex $$ etex);
            draw marquesegment(A, B, 1);
            draw marquesegment(C, B, 1);
            draw marqueangle(B, C, A, 1);
            draw marqueangle(C, A, B, 1);
          fin;
        \end{mplibcode}
        &
        \begin{mplibcode}
          u = 0.5cm;
          taille_marque_s := 0.15cm;
          angle_marque_s := 90;
          repere(-1,5,u,-1,5,u);
            pair A, B, C;
            A := origin;
            B := 4.2 * dir(65);
            C := (3, 0);
            draw triangle(A, B, C) withpen pencircle scaled 1.25;
            nomme.top(A, btex $$ etex);
            nomme.top(B, btex $$ etex);
            nomme.top(C, btex $$ etex);
          fin;
        \end{mplibcode} \\
        Equilateral Triangle & Isosceles Triangle & Scalene \\
        All sides and angles are equal & One pair of sides and angles equal & No sides or angles equal
      \end{tabularx}

    \end{column}
    \begin{column}{0.5\textwidth}
      \begin{definition}
        The sum of the angles in a triangle is always $180°$. 
      \end{definition}
      \begin{center}
        \leavevmode
        \begin{mplibcode}
          u = 0.8cm;
          taille_marque_s := 0.15cm;
          angle_marque_s := 90;
          repere(-1,6,u,-1,6,u);
            pair A, B, C, D, E;
            A := origin;
            B := 3 * dir(45);
            C := (5, 0);
            E := B + 2 * dir(180);
            D := B + 2 * dir(0);
            draw (0, ypart B)--(5, ypart B) withpen pencircle scaled 1.25 withcolor vertfonce;
            draw triangle(A, B, C) withpen pencircle scaled 1.25;
            nomme.llft(A, btex $A$ etex);
            nomme.top(B, btex $B$ etex);
            nomme.lrt(C, btex $C$ etex);
            nomme.top(E, btex $D$ etex);
            nomme.top(D, btex $E$ etex);
            label.bot(btex $\gamma$ etex, B + 0.2 * dir(-90));
            label.llft(btex $\alpha$ etex, B + 0.5 * dir(180));
            label.lrt(btex $\beta$ etex, B + 0.8 * dir(0));
            label.urt(btex $\beta$ etex, A + 0.6 * dir(0));
            label.ulft(btex $\alpha$ etex, C + 0.6 * dir(180));
          fin;
        \end{mplibcode}
      \end{center}

      Proof: Draw a line parallel to side $AB$ of triangle $ABC$ passing through point $C$.  As alternate interior angles, we have the pairs of angles labeled $\alpha$ and $\beta$ in the figure equal.  Since they form a line $ED$, $\angle ACB + \angle ECA + \angle BCD = 180°$.  This means $\gamma + \alpha + \beta = 180°$.  Q.E.D.
    \end{column}
  \end{columns}
\end{frame}

% \begin{frame}{Title}
%   \begin{columns}[T]
%     \begin{column}{0.5\textwidth}
%     \end{column}
%     \begin{column}{0.5\textwidth}
%     \end{column}
%   \end{columns}
% \end{frame}

% \begin{frame}{Title}
%   \begin{columns}[T]
%     \begin{column}{0.5\textwidth}
%     \end{column}
%     \begin{column}{0.5\textwidth}
%     \end{column}
%   \end{columns}
% \end{frame}

% \begin{frame}{Title}
%   \begin{columns}[T]
%     \begin{column}{0.5\textwidth}
%     \end{column}
%     \begin{column}{0.5\textwidth}
%     \end{column}
%   \end{columns}
% \end{frame}

% \begin{frame}{Title}
%   \begin{columns}[T]
%     \begin{column}{0.5\textwidth}
%     \end{column}
%     \begin{column}{0.5\textwidth}
%     \end{column}
%   \end{columns}
% \end{frame}

% \begin{frame}{Title}
%   \begin{columns}[T]
%     \begin{column}{0.5\textwidth}
%     \end{column}
%     \begin{column}{0.5\textwidth}
%     \end{column}
%   \end{columns}
% \end{frame}

% \begin{frame}{Title}
%   \begin{columns}[T]
%     \begin{column}{0.5\textwidth}
%     \end{column}
%     \begin{column}{0.5\textwidth}
%     \end{column}
%   \end{columns}
% \end{frame}

% \begin{frame}{Title}
%   \begin{columns}[T]
%     \begin{column}{0.5\textwidth}
%     \end{column}
%     \begin{column}{0.5\textwidth}
%     \end{column}
%   \end{columns}
% \end{frame}

% \begin{frame}{Title}
%   \begin{columns}[T]
%     \begin{column}{0.5\textwidth}
%     \end{column}
%     \begin{column}{0.5\textwidth}
%     \end{column}
%   \end{columns}
% \end{frame}

% \begin{frame}{Title}
%   \begin{columns}[T]
%     \begin{column}{0.5\textwidth}
%     \end{column}
%     \begin{column}{0.5\textwidth}
%     \end{column}
%   \end{columns}
% \end{frame}


% \begin{frame}{Title}
%   \begin{columns}[T]
%     \begin{column}{0.5\textwidth}
%     \end{column}
%     \begin{column}{0.5\textwidth}
%     \end{column}
%   \end{columns}
% \end{frame}

\end{document}