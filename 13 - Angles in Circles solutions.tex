\documentclass[9pt,aspectratio=169,handout]{beamer}

\usepackage{scalerel}

\usetheme{graham}

\title{Angles in Circles solutions}
\subtitle[Graham Middle School]{Graham Middle School Math Olympiad Team}

\newcommand\Mydiv[2]{%
$\strut#1$\kern.25em\smash{\raise.3ex\hbox{$\big)$}}$\mkern-8mu
        \overline{\enspace\strut#2}$}

\setcounter{MaxMatrixCols}{20}
\newcommand{\Mod}[1]{\ (\mathrm{mod}\ #1)}
\newcommand{\longdiv}{\smash{\mkern-0.43mu\vstretch{1.31}{\hstretch{.7}{)}}\mkern-5.2mu\vstretch{1.31}{\hstretch{.7}{)}}}}

\DeclareMathOperator{\lcm}{lcm}

\usepackage{luamplib}
\mplibsetformat{metafun}
\mplibtextextlabel{enable}
\everymplib{input mpcolornames; input repere; input macros; beginfig(1);}
\everyendmplib{endfig;}

\begin{document}
\maketitle

\begin{frame}{Exercises}
  \begin{columns}[T]
    \begin{column}{0.5\textwidth}
      \begin{columns}[T, totalwidth=\textwidth]
        \begin{column}{0.6\linewidth}
          % \setlength{\leftmargini}{0.2cm}
          \begin{enumerate}
            % \conti
            \item Point $O$ is the center of the regular octagon $ABCDEFGH$, and $X$ is the midpoint of the side $\overline{AB}.$ What fraction of the area of the octagon is shaded? % AMC 8 2015, Problem 2
            \item The circumference of the circle with center $O$ is divided into 12 equal arcs, marked the letters $A$ through $L$ as seen below. What is the number of degrees in the sum of the angles $x$ and $y$? %AMC 8 2014, Problem 15
            \seti
          \end{enumerate}
        \end{column}
        \begin{column}{0.4\linewidth}
          % \vspace*{-1em}
          \leavevmode
          \begin{mplibcode}
            u = 1cm;
            path c;
            pair A, B, C, D, E, F, G, H, X, O;
            O := origin;
            c := circle(origin, 1u);
            A := point 0.5 of c;
            B := point 1 of c;
            C := point 1.5 of c;
            D := point 2 of c;
            E := point 2.5 of c;
            F := point 3 of c;
            G := point 3.5 of c;
            H := point 0 of c;
            X := 0.5[A, B];
            fill X--B--C--D--O--cycle withcolor 0.7white;
            Draw A--B--C--D--E--F--G--H--cycle, D--O--X;
            Dot A, B, C, D, E, F, G, H, X, O;
            label.urt("$A$", A);
            label.top("$B$", B);
            label.ulft("$C$", C);
            label.lft("$D$", D);
            label.llft("$E$", E);
            label.bot("$F$", F);
            label.lrt("$G$", G);
            label.rt("$H$", H);
            label.urt("$X$", X);
            label.lrt("$O$", O);
          \end{mplibcode}

          \vspace{10pt}
          \begin{mplibcode}
            u = 1cm;
            path c;
            pair A, B, C, D, E, F, G, H, I, J, K, L, O;
            O := origin;
            c := circle(origin, 1u);
            A := point 4/3 of c;
            B := point 3/3 of c;
            C := point 2/3 of c;
            D := point 1/3 of c;
            E := point 0 of c;
            F := point 11/3 of c;
            G := point 10/3 of c;
            H := point 9/3 of c;
            I := point 8/3 of c;
            J := point 7/3 of c;
            K := point 6/3 of c;
            L := point 5/3 of c;
            Draw c, A--D--O--cycle, O--G--I--cycle;
            Dot A, B, C, D, E, F, G, H, I, J, K, L, O;
            labelarcsprof(O, A, D, 10, 8, "$x$");
            labelarcsprof2(O, G, I, 10, 7, "$y$");
            label.ulft("$A$", A);
            label.top("$B$", B);
            label.urt("$C$", C);
            label.urt("$D$", D);
            label.rt("$E$", E);
            label.lrt("$F$", F);
            label.lrt("$G$", G);
            label.bot("$H$", H);
            label.llft("$I$", I);
            label.llft("$J$", J);
            label.lft("$K$", K);
            label.ulft("$L$", L);
            label.lft("$O$", O);
          \end{mplibcode}
        \end{column}
      \end{columns}
      \begin{enumerate}
        \conti
        \item What is the angle (less than $180°$) formed by the hands of a clock when the time is $3{:}15$?
        \seti
      \end{enumerate}
    \end{column}
    \begin{column}{0.5\textwidth}
      \setlength{\leftmargini}{0.2cm}
      \begin{enumerate}
        \conti
        \item Points $A$, $B$, $C$, and $D$ are on a circle with center $O$ in that order, and $AC$ and $BD$ meet at $E$. Given that $\angle AOB = 34°$,
        $\angle COD = 102°$, $\angle DOA =109°$, find $\angle BEC$.
        \item Points $A$, $B$, $C$, and $D$ are on a circle with center $O$ in that order, and $DA$ and $CB$ meet at $E$. Given that $\angle AOB = 34°$,
        $\angle COD = 102°$, $\angle DOA =109°$, find $\angle AEB$.
        \item Chords $\overline{TY}$ and $\overline{OP}$ meet at point $K$ such that $TK = 2$, $KY = 16$, and $KP = 2\cdot KO$. Find $OP$.
        \item Points $A$, $B$, $C$, and $D$ are on a circle in that order and $AB$ intersects $DC$ in point $P$ outside the circle. We have $BP = 8$, $AB = 10$, $CD = 7$, and $\angle APC = 60°$. Find the radius of the circle.
        \item Diameter $AB$ of a circle has length $25$.  Point $C$ is chosen along the circumference such that $AC$ has length $24$.  What is the length of $BC$?     
      \end{enumerate}
    \end{column}
  \end{columns}
\end{frame}

\begin{frame}{Challenge problems}
  \begin{columns}[T]
    \begin{column}{0.5\textwidth}
      \begin{enumerate}
        \item Let the incircle of triangle $ABC$ be tangent to sides $BC$, $AC$, and $AB$ at points $D$, $E$, and $F$, respectively. Given that $\angle A = 32°$, find $\angle EDF$.
        \item The areas of two adjacent squares are $256$ square inches and $16$ square inches,
        respectively, and their bases lie on the same line. What is the number of inches in
        the length of the segment that joins the centers of the two inscribed circles? 
        \item We are given points $A$, $B$, $C$, and $D$ in the plane such that $AD = 13$ while $AB = BC = AC= CD = 10$. Find $\angle ADB$.
        \item In $\triangle ABC$, $AB = 86$, and $AC=97$. A circle with center $A$ and radius $AB$ intersects $\overline{BC}$ at points $B$ and $X$. Moreover $\overline{BX}$ and $\overline{CX}$ have integer lengths. What is $BC$?
      \end{enumerate}
    \end{column}
    \begin{column}{0.5\textwidth}
    \end{column}
  \end{columns}
\end{frame}

\begin{frame}{Exercises 1-2}
  \begin{columns}[T]
    \begin{column}{0.5\textwidth}
      \begin{problem}
        \textbf{E1.} Point $O$ is the center of the regular octagon $ABCDEFGH$, and $X$ is the midpoint of the side $\overline{AB}.$ What fraction of the area of the octagon is shaded?
      \end{problem}
      \begin{wrapfigure}{l}{20.00mm}
        \begin{mplibcode}
          u = 1cm;
            path c;
            pair A, B, C, D, E, F, G, H, X, O;
            O := origin;
            c := circle(origin, 1u);
            A := point 0.5 of c;
            B := point 1 of c;
            C := point 1.5 of c;
            D := point 2 of c;
            E := point 2.5 of c;
            F := point 3 of c;
            G := point 3.5 of c;
            H := point 0 of c;
            X := 0.5[A, B];
            fill X--B--C--D--O--cycle withcolor 0.7white;
            Draw A--B--C--D--E--F--G--H--cycle, D--O--X;
            draw A--O--B dashed evenly;
            draw C--O dashed evenly;
            Dot A, B, C, D, E, F, G, H, X, O;
            label.urt("$A$", A);
            label.top("$B$", B);
            label.ulft("$C$", C);
            label.lft("$D$", D);
            label.llft("$E$", E);
            label.bot("$F$", F);
            label.lrt("$G$", G);
            label.rt("$H$", H);
            label.urt("$X$", X);
            label.lrt("$O$", O);
        \end{mplibcode}
      \end{wrapfigure}
      Area of each triangle $YOZ$ where $Y$ and $Z$ are neighbor points from $A$ to $H$ is $1/8$ of the area of the octagon. Since $DCBXO$ contains $2.5$ of such triangles, so the total fraction is $2.5 \cdot \dfrac{1}{8} = \dfrac{5}{2} \cdot \dfrac{1}{8} = \boxed{\dfrac{5}{16}}$. 
    \end{column}
    \begin{column}{0.5\textwidth}
      \begin{problem}
        \textbf{E2.} The circumference of the circle with center $O$ is divided into 12 equal arcs, marked the letters $A$ through $L$ as seen below. What is the number of degrees in the sum of the angles $x$ and $y$?
      \end{problem}
      \begin{wrapfigure}{l}{20.00mm}
        \begin{mplibcode}
          u = 1cm;
          path c;
          pair A, B, C, D, E, F, G, H, I, J, K, L, O;
          O := origin;
          c := circle(origin, 1u);
          A := point 4/3 of c;
          B := point 3/3 of c;
          C := point 2/3 of c;
          D := point 1/3 of c;
          E := point 0 of c;
          F := point 11/3 of c;
          G := point 10/3 of c;
          H := point 9/3 of c;
          I := point 8/3 of c;
          J := point 7/3 of c;
          K := point 6/3 of c;
          L := point 5/3 of c;
          Draw c, A--D--O--cycle, O--G--I--cycle;
          Dot A, B, C, D, E, F, G, H, I, J, K, L, O;
          labelarcsprof(O, A, D, 10, 8, "$x$");
          labelarcsprof2(O, G, I, 10, 7, "$y$");
          label.ulft("$A$", A);
          label.top("$B$", B);
          label.urt("$C$", C);
          label.urt("$D$", D);
          label.rt("$E$", E);
          label.lrt("$F$", F);
          label.lrt("$G$", G);
          label.bot("$H$", H);
          label.llft("$I$", I);
          label.llft("$J$", J);
          label.lft("$K$", K);
          label.ulft("$L$", L);
          label.lft("$O$", O);
        \end{mplibcode}
      \end{wrapfigure} 
      $\angle AOD = \cfrac{360°}{12}\cdot 3 = 90°$, and since $AO = OD$, $\angle OAD = \angle ADO = x$. So $2x + 90° = 180°$ and $x = 45°$. In similar way we got $\angle IOG = \cfrac{360°}{12} \cdot 2 = 60°$ and $2y + 60° = 180°$, so $y = 60°$. $x + y = 45° + 60° = \boxed{105°}$. 
    \end{column}
  \end{columns}
\end{frame}

\begin{frame}{Exercises 3-5}
  \begin{columns}[T]
    \begin{column}{0.5\textwidth}
      \begin{problem}
        \textbf{E3.} What is the angle (less than $180°$) formed by the hands of a clock when the time is $3{:}15$?
      \end{problem}
      \begin{wrapfigure}{l}{20.00mm}
        \vspace*{-\intextsep}
        \begin{mplibcode}
          r := 1.25cm; len := 3bp; min_thickness := 1bp; hour_thickness := 2bp;
          path cadran; cadran = fullcircle scaled (2r); 
          fill cadran scaled 1.04; fill cadran withcolor .75[rgb_orange, white]; 
          % Graduation marks and labels
          for i = 1 upto 60:
            if i mod 5 = 0: 
              j := i div 5; angl := 90-30j; 
              freelabel("\sffamily\bfseries" & decimal j, (r-len)*dir angl, r*dir angl);
              draw ((r, 0) -- (r - len, 0)) rotated angl withpen pencircle scaled min_thickness;
            else:
              angl := 90 - 6i;
              draw ((r, 0) -- (r - .5len, 0)) rotated angl;
            fi
          endfor
          % Needles
          minute := 15;
          hour := 3;
          pickup pencircle scaled min_thickness;
          drawarrow origin -- r*dir(90-minute*6) cutends (0, 4len);
          pickup pencircle scaled hour_thickness;
          drawarrow origin -- r*dir (90-(hour+minute/60)*30) cutends(0, 6len);
          fill fullcircle scaled 2len;
        \end{mplibcode}
        \vspace*{-\intextsep}
      \end{wrapfigure}
      The minute hand moved $15 \cdot \cfrac{360°}{60} = 90°$ from the top. 
      \begin{wrapfigure}{r}{20.00mm}
        \vspace*{-2\intextsep}
      \end{wrapfigure}
      The hours hand moved $3.25 \cdot \cfrac{360°}{12} = \cfrac{13}{4} \cdot 30° = 97.5°$. So the angle is $97.5° - 90° = \boxed{7.5°}$.

      \begin{problem}
        \textbf{E4.} Points $A$, $B$, $C$, and $D$ are on a circle with center $O$ in that order, and $AC$ and $BD$ meet at $E$. Given that $\angle AOB = 34°$,
        $\angle COD = 102°$, $\angle DOA =109°$, find $\angle BEC$.
      \end{problem}

      \begin{problem}
        \textbf{E5.} Points $A$, $B$, $C$, and $D$ are on a circle with center $O$ in that order, and $DA$ and $CB$ meet at $E'$. Given that $\angle AOB = 34°$,
        $\angle COD = 102°$, $\angle DOA =109°$, find $\angle AE'B$.
      \end{problem}
    \end{column}
    \begin{column}{0.55\textwidth}
      \begin{center}
        \vspace*{-2em}
        \leavevmode
        \begin{mplibcode}
          u = 1cm;
          path c;
          pair A, B, C, D, E, Ei;
          c := circle(origin, 1u);
          A := point 0 of c;
          B := point 34/90 of c;
          D := point -109/90 of c;
          C := point -211/90 of c;
          E := whatever[A, C]=whatever[B, D];
          Ei := whatever[D, A]=whatever[C, B];
          Draw c, A--C, B--D;
          draw D--Ei--C dashed evenly;
          Dot A, B, C, D, E, Ei;
          label.lrt("$A$", A);
          label.urt("$B$", B);
          label.ulft("$C$", C);
          label.llft("$D$", D);
          label.bot("$E$", E shifted (1, -1));
          label.top("$E'$", Ei);
        \end{mplibcode}
      \end{center}
      Problem 4:
  
      $\angle ACB = \frac{1}{2} \angle AOB = 17°$ and $\angle DBC = \frac{1}{2} \angle COD = 51°$. So $\angle BEC = 180° - \angle ECB - \angle EBC = 180° - 17° - 51° = \boxed{112°}$. In general $\angle BEC = \dfrac{\angle CEB + \angle DEA}{2}$.

      Problem 5:
  
      $\angle CAD = \frac{1}{2} \angle COB = 51°$ and $\angle ACB = \frac{1}{2} \angle AOB = 17°$. So $\angle CAE' = 180° - \angle CAD = 129°$, and $\angle AE'B = 180° - 17° - 129° = \boxed{34°}$. 
      
      In general $\angle AE'B = \dfrac{\angle COD - \angle AOB}{2}$.
    \end{column}
  \end{columns}
\end{frame}

\begin{frame}{Exercises 6-8}
  \begin{columns}[T]
    \begin{column}{0.55\textwidth}
      \begin{problem}
        \textbf{E6.} Chords $\overline{TY}$ and $\overline{OP}$ meet at point $K$ such that $TK = 2$, $KY = 16$, and $KP = 2\cdot KO$. Find $OP$.
      \end{problem}
      \begin{wrapfigure}{r}{20.00mm}
        \vspace*{-1.2\intextsep}
        \begin{mplibcode}
          u = 0.7cm;
          path c;
          pair T, Y, O, P, K;
          c := circle(origin, 1u);
          T := point 0 of c;
          Y := point 1.5 of c;
          P := point 0.5 of c;
          O := point 2.5 of c;
          K := whatever[O, P]=whatever[T, Y];
          Draw c, T--Y, O--P;
          Dot T, Y, O, P, K;
          label.rt("$T$", T);
          label.ulft("$Y$", Y);
          label.urt("$P$", P);
          label.top("$O$", O shifted (0, 1));
          label.bot("$K$", K);
        \end{mplibcode}
        \vspace*{-3\intextsep}
      \end{wrapfigure}
      Let $KO = x$, than $KP = 2x$ and $OP = 3x$. Since $KP \cdot KO = TK \cdot KY$, $2x^2 = 32$ and $x = 4$. So $OP = \boxed{12}$.
      \begin{problem}
        \textbf{E7.} Points $A$, $B$, $C$, and $D$ are on a circle in that order and $AB$ intersects $DC$ in point $P$ outside the circle. We have $BP = 8$, $AB = 10$, $CD = 7$, and $\angle APC = 60°$. Find the radius of the circle.
      \end{problem}

      \begin{wrapfigure}{l}{20.00mm}
        \vspace*{-\intextsep}
        \begin{mplibcode}
          u = 0.7cm;
          path c;
          pair A, B, C, D, P, O;
          c := circle(origin, 1u);
          A := point 0.2 of c;
          B := point 1 of c;
          C := point 1.6 of c;
          D := point 2.3 of c;
          P := whatever[A, B]=whatever[C, D];
          Draw c, A--P--D;
          draw D--A--C dashed evenly;
          Dot A, B, C, D, P;
          label.rt("$A$", A);
          label.urt("$B$", B);
          label.ulft("$C$", C);
          label.llft("$D$", D);
          label.top("$P$", P);
        \end{mplibcode}
        \vspace*{-\intextsep}
      \end{wrapfigure}
      $PC \cdot PD = PA \cdot PB$, so if $PC = x$, $x (x + 7) = 8 \cdot 18$ and $x = 9$ or $-16$. The later root doesn't make sense, so $PC = 9$. 
      So $\triangle PCA$ is $30-60-90$ triangle and $\angle ACD = 90°$. So $DA$ is the diameter of the circle and $DA = \sqrt{DC^2 + CA^2} = \sqrt{DC^2 + (\sqrt{3} PC)^2}=\sqrt{49 + 3\cdot 81} = \sqrt{292} = 2\cdot\sqrt{73}$. So radius is $\boxed{\sqrt{73}}$.
      
   \end{column}
    \begin{column}{0.45\textwidth}
      \begin{problem}
        \textbf{E8.} Diameter $AB$ of a circle has length $25$.  Point $C$ is chosen along the circumference such that $AC$ has length $24$.  What is the length of $BC$? 
      \end{problem}
      \begin{wrapfigure}{r}{20.00mm}
        \begin{mplibcode}
          u = 1cm;
          path c;
          pair A, B, C;
          c := circle(origin, 1u);
          A := point 0 of c;
          B := point 2 of c;
          C := point 1.4 of c;
          Draw c, A--B--C--cycle;
          mark_rt_angle(B, C, A);
          Dot A, B, C;
          label.rt("$A$", A);
          label.lft("$B$", B);
          label.ulft("$C$", C);
        \end{mplibcode}
      \end{wrapfigure}
      Since $AB$ is diameter, $\angle C = 90°$. And using the Pythagorean theorem $BC^2 = AB^2 - AC^2$. So $BC = \sqrt{25^2 - 24^2} = \sqrt{(25 + 24) (25- 24)} = \sqrt{49} = \boxed{7}$.
    \end{column}
  \end{columns}
\end{frame}

\begin{frame}{Challenge problems 1-2}
  \begin{columns}[T]
    \begin{column}{0.5\textwidth}
      \begin{problem}
        \textbf{CP1.} Let the incircle of triangle $ABC$ be tangent to sides $BC$, $AC$, and $AB$ at points $D$, $E$, and $F$, respectively. Given that $\angle A = 32°$, find $\angle EDF$.
      \end{problem}
      \begin{center}
        \leavevmode
        \begin{mplibcode}
          u = 1.3cm;
          pair A, B, C, D, E, F, I;
          path c;
          A := origin;
          B := 3u*dir(0);
          C := 2.5u*dir(32);
          I := incenter(A, B, C);
          c := incircle(A, B, C);
          D := altitude(B, I, C);
          E := altitude(A, I, C);
          F := altitude(B, I, A);
          Draw A--B--C--cycle, c;
          draw E--D--F--I--cycle dashed evenly;
          Dot A, B, C, D, E, F, I;
          label.bot("$A$", A);
          label.bot("$B$", B);
          label.top("$C$", C);
          label.urt("$D$", D);
          label.ulft("$E$", E);
          label.bot("$F$", F);
          label.lft("$I$", I shifted (-1, -2));
        \end{mplibcode}
      \end{center}
      $\angle IEA = \angle IFA = 90°$, so $\angle EIF = 360° - 90° - 90° - 32° = 148°$. And $\angle EDF = \frac{1}{2} \angle EIF = \boxed{74°}$.
    \end{column}
    \begin{column}{0.5\textwidth}
      \begin{problem}
        \textbf{CP2.} The areas of two adjacent squares are $256$ square inches and $16$ square inches,
        respectively, and their bases lie on the same line. What is the number of inches in
        the length of the segment that joins the centers of the two inscribed circles? 
      \end{problem}
      \begin{wrapfigure}{r}{20.00mm}
        \begin{mplibcode}
          u = 1cm;
          Draw (1u, 0u)--(0u, 0u)--(0u, 1u)--(1u, 1u), (1u, 0u)--(4u, 0u)--(4u, 3u)--(1u, 3u)--cycle, (0.5u, 0.5u)--(2.5u, 1.5u);
          draw (0.5u, 0.5u)--(2.5u, 0.5u)--(2.5u, 1.5u) dashed evenly;
          Dot (0.5u, 0.5u), (2.5u, 1.5u);
          mark_rt_angle((0.5u, 0.5u), (2.5u, 0.5u), (2.5u, 1.5u));
        \end{mplibcode}
      \end{wrapfigure}
      The sides of both squares are $\sqrt{16} = 4$ and $\sqrt{256} = 16$. So horizontal distance between centers is $4/2 + 16/2 = 10$ inches, and vertical distance is $16/2 - 4/2 = 6$ inches. So distance between the points is $\sqrt{10^2 + 6^2} = \sqrt{136} = \boxed{2 \sqrt{34}}$.
    \end{column}
  \end{columns}
\end{frame}

\begin{frame}{Challenge problems 3-4}
  \begin{columns}[T]
    \begin{column}{0.5\textwidth}
      \begin{problem}
        \textbf{CP3.} We are given points $A$, $B$, $C$, and $D$ in the plane such that $AD = 13$ while $AB = BC = AC= CD = 10$. Find $\angle ADB$.
      \end{problem}
      \begin{wrapfigure}{r}{20.00mm}
        \begin{mplibcode}
          u = 1cm;
          pair A, B, C, D;
          path c;
          c := circle(origin, 1.5u);
          C := origin;
          A := point 0 of c;
          B := point 2/3 of c;
          D := point -0.9 of c;
          Draw c, A--B--D--cycle, A--C--D, B--C;
          Dot A, B, C, D;
          label.rt("$A$", A);
          label.urt("$B$", B);
          label.lft("$C$", C);
          label.bot("$D$", D);
        \end{mplibcode}
      \end{wrapfigure}
      Since $A$, $B$, and $C$ all equidistance from $C$, points $A$, $B$, and $D$ are all on the circle with the center $C$. Moreover, $\triangle ABC$ is equilateral. So $\angle ADB = \frac{1}{2} \angle BCA = \boxed{30°}$. 
    \end{column}
    \begin{column}{0.5\textwidth}
      \begin{problem}
        \textbf{CP4.} In $\triangle ABC$, $AB = 86$, and $AC=97$. A circle with center $A$ and radius $AB$ intersects $\overline{BC}$ at points $B$ and $X$. Moreover $\overline{BX}$ and $\overline{CX}$ have integer lengths. What is $BC$?
      \end{problem}
      \begin{wrapfigure}{r}{20.00mm}
        \vspace*{-\intextsep}
        \begin{mplibcode}
          u = 1.6cm;
          pair A, B, C, X, Y, Z;
          path c;
          c := circle(origin, 1u);
          A := origin;
          B := point 0 of c;
          C := cross.top(circle(A, 1u*97/86), circle(B, 1u*61/86));
          X := cross.top(c, B--C);
          Y := cross.top(c, C--A);
          Z := Y rotatedaround (A, 180);
          Draw c, A--B--C--cycle;
          draw A--Z dashed evenly;
          Dot A, B, C, X, Y, Z;
          label.ulft("$A$", A);
          label.rt("$B$", B);
          label.urt("$C$", C);
          label.rt("$X$", X);
          label.lft("$Y$", Y shifted (-3, 0));
          label.llft("$Z$", Z);
        \end{mplibcode}
        \vspace*{-\intextsep}
      \end{wrapfigure}
      Let $BX = q$, $CX = p$, and $AC$ meets the circle at $Y$ and $Z$, with $Y$ on $AC$. Then $AZ = AY = 86$. Using the Power of a Point, we get that $p(p+q) = 11(183) = 11 \cdot 3 \cdot 61$. We know that $p+q>p$, so $p$ is either $3$, $11$, or $33$. We also know that $p>11$ by the triangle inequality on $\triangle ACX$. Thus, $p$ is $33$ so we get that $BC = p+q = \boxed{61}$.
    \end{column}
  \end{columns}
\end{frame}

% \begin{frame}{Title}
%   \begin{columns}[T]
%     \begin{column}{0.5\textwidth}
%     \end{column}
%     \begin{column}{0.5\textwidth}
%     \end{column}
%   \end{columns}
% \end{frame}

% \begin{frame}{Title}
%   \begin{columns}[T]
%     \begin{column}{0.5\textwidth}
%     \end{column}
%     \begin{column}{0.5\textwidth}
%     \end{column}
%   \end{columns}
% \end{frame}

\end{document}