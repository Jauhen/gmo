\documentclass[9pt,aspectratio=169,handout]{beamer}

\usepackage{scalerel}

\usetheme{graham}

\title{Quadratic Equations solutions}
\subtitle[Graham Middle School]{Graham Middle School Math Olympiad Team}

\newcommand\Mydiv[2]{%
$\strut#1$\kern.25em\smash{\raise.3ex\hbox{$\big)$}}$\mkern-8mu
        \overline{\enspace\strut#2}$}

\setcounter{MaxMatrixCols}{20}
\newcommand{\Mod}[1]{\ (\mathrm{mod}\ #1)}
\newcommand{\longdiv}{\smash{\mkern-0.43mu\vstretch{1.31}{\hstretch{.7}{)}}\mkern-5.2mu\vstretch{1.31}{\hstretch{.7}{)}}}}

\DeclareMathOperator{\lcm}{lcm}

\usepackage{luamplib}
\mplibsetformat{metafun}
\mplibtextextlabel{enable}
\everymplib{input repere; input macros; beginfig(1);}
\everyendmplib{endfig;}

\begin{document}
\maketitle

\begin{frame}{Exercises}
  \begin{columns}[T]
    \begin{column}{0.5\textwidth}
      \begin{enumerate}
        \item Use factoring to find the roots of $x^2 – 22x – 48 = 0$.
        \item Complete the square to find the possible values of $x$ for which $x^2 + 4x + 3 = 0$?
        \item What is the value of $i^3$?
        \item For what value(s) of $x$ does the fraction of $3$ raised to the power $x^2$ over $3$ raised to the power $3x$ equal one-ninth?  
        
        (\emph{Hint}: If all exponents have the same base, then we can solve the problem by equating the exponents.
        \item Find the roots of $x = \dfrac{28}{x - 3}$.
        \item If $b$ and $c$ are both rational numbers and one of the roots of $x^2 + bx + c = 0$ is $3 + \sqrt{2}$, find $b$ and $c$.
        \seti
      \end{enumerate}
    \end{column}
    \begin{column}{0.5\textwidth}
      \begin{enumerate}
        \conti
        \item For how many different integer values of $b$ are both roots of $x^2 + bx - 16 = 0$ integers?
        \item Let $m$ and $n$ be roots of: $x^2 − 60x + 864 = 0$.  Find a polynomial with roots $m + 1$ and $n + 1$.
      \end{enumerate}
    \end{column}
  \end{columns}
\end{frame}

\begin{frame}{Challenge problems}
  \begin{columns}[T]
    \begin{column}{0.5\textwidth}
      \begin{enumerate}
        \item Find the minimum possible value of the absolute value of $(m - n)$, where $m$ and $n$ are integers satisfying $m + n = mn – 2021$. 
        
        (\emph{Hint}: could completing the square be useful here if the variables were all grouped on one side of the equation?)
        \item (For fun) In the novel, “The Curious Incident of the Dog in the Nighttime,” a student in England taking his A-level college entrance exam in maths was given the following question: 
        
        Prove that a triangle with sides that can written in the form $n^2 +1$, $n^2 -1$ and $2n$ (where $n > 1$) is right-angled.  
        \seti
      \end{enumerate}
    \end{column}
    \begin{column}{0.5\textwidth}
      \begin{enumerate}
        \conti
        \item Let $m$ and $n$ be roots of the polynomial $x^2 − 60x + 899 = 0.$  What is $m^2 + n^2$? 
        
        (\emph{Hint}: think about how $m^2 + n^2$ can be rewritten in terms of the sum and product of the roots $m$ and $n$).
        \item Find all real values of $n$ such that $2^{2n} + 2^n + 1 = 73$.  
        
        (\emph{Hint}:  What substitution would turn this into a quadratic?)
      \end{enumerate}
    \end{column}
  \end{columns}
\end{frame}

\begin{frame}{Exercises 1-4}
  \begin{columns}[T]
    \begin{column}{0.5\textwidth}
      \begin{problem}
        \textbf{E1.} Use factoring to find the roots of 
        
        $x^2 – 22x – 48 = 0$.
      \end{problem}
      Let $x^2 - 22x - 48 = (x - a)(x - b)$, so $a + b = -22$ and $ab = -48$. Looking into divisors of $-48$ we may find that $-24$ and $2$ has a sum of $-22$, so the root of the equation are $\boxed{-22}$ and $\boxed{2}$. 

      \begin{problem}
        \textbf{E2.} Complete the square to find the possible values of $x$ for which $x^2 + 4x + 3 = 0$?
      \end{problem}
      To complete square for $x^2 + 4x$ we need to add $4$, since $(x+2)^2 = x^2 + 4x + 4$. So, in our equation we have $x^2 + 4x + 3 + 1 = 1$ or $(x+2)^2 = 1$. This gives us $x + 2 = \pm 1$ and $x = \boxed{-3}$ or $\boxed{-1}$.

    \end{column}
    \begin{column}{0.5\textwidth}
      \begin{problem}
        \textbf{E3.} What is the value of $i^3$?
      \end{problem}
      $i^3 = (i)^2 \times i = -1 \times i = -i$.
      \begin{problem}
        \textbf{E4.} For what value(s) of $x$ does the fraction of $3$ raised to the power $x^2$ over $3$ raised to the power $3x$ equal one-ninth?  
        
        (\emph{Hint}: If all exponents have the same base, then we can solve the problem by equating the exponents.
      \end{problem}
      We need to solve this equation:
      \[ \frac{3^{x^2}}{3^{3x}} = \frac{1}{9} = \frac{1}{3^2}.\]
      Which can be rewritten as $3^{x^2} \times 3^{-3x} = 3^{-2}$, or, using our hint, $x^2 - 3x = -2$, $x^2 - 3x + 2 = (x - 1)(x - 2) = 0$ and our equation has two roots $x = \boxed{1}$ and $\boxed{2}$. Plugging them back into our original equation we can check that they are works.
    \end{column}
  \end{columns}
\end{frame}

\begin{frame}{Exercises 5-8}
  \begin{columns}[T]
    \begin{column}{0.5\textwidth}
      \begin{problem}
        \textbf{E5.} Find the roots of $x = \dfrac{28}{x - 3}$.
      \end{problem}
      $x(x - 3) = 28$ or $x^2 - 3x - 28 = (x - 7)(x + 4) = 0$ and two roots $x = \boxed{-4}$ and $\boxed{7}$. Plugging them back into our original equation we can check that they are works.
      \begin{problem}
        \textbf{E6.} If $b$ and $c$ are both rational numbers and one of the roots of $x^2 + bx + c = 0$ is $3 + \sqrt{2}$, find $b$ and~$c$.
      \end{problem}
      The roots are $\dfrac{-b}{2} \pm \sqrt{\dfrac{b^2 - 4c}{4}}$, so we may see that rational and irrational parts of both roots should be the same and the second root is $3 - \sqrt{2}$. $x^2 + bx + c = (x - (3 +\sqrt{2}))(x - (3 - \sqrt{2})) = x^2 - 6x + 7 = 0$. So $b = -6$ and $c = 7$.
    \end{column}
    \begin{column}{0.5\textwidth}
      \begin{problem}
        \textbf{E7.} For how many different integer values of $b$ are both roots of $x^2 + bx - 16 = 0$ integers?
      \end{problem}
      Let $x^2 + bx - 16 = (x + p)(x + q) = 0$, so $pq = -16$ and $b = p + q$. Since $-16 = 1\cdot -16 = 2\cdot -8 = 4\cdot -4 = 8\cdot -2 = 16\cdot -1 = -1\cdot 16= -2\cdot 8= -4\cdot 4= -8\cdot 2= -16\cdot 1$. We got $\boxed{5}$ different values for $b$: $-15$, $-6$, $0$, $6$, and $15$.
      \begin{problem}
        \textbf{E8.} Let $m$ and $n$ be roots of: $x^2 - 60x + 864 = 0$.  Find a polynomial with roots $m + 1$ and $n + 1$.
      \end{problem}
      $(x - (m + 1))(x - (n +1)) = x^2 - (m + n + 2)x + (mn + m + n + 1) = 0$ has roots $m +1$ and $n +1$, since $m + n = 60$ and $mn = 864$, we got $\boxed{x^2 - 62x + 925 = 0}$.
    \end{column}
  \end{columns}
\end{frame}

\begin{frame}{Challenge problems 1 - 2}
  \begin{columns}[T]
    \begin{column}{0.5\textwidth}
      \begin{problem}
        \textbf{CP1.} Find the minimum possible value of the absolute value of $(m - n)$, where $m$ and $n$ are integers satisfying $m + n = mn - 2021$. 
        
        (\emph{Hint}: could completing the square be useful here if the variables were all grouped on one side of the equation?)
      \end{problem}
      $mn - m - n - 2021 = (m - 1)(n - 1) - 2022$ or $(m - 1)(n - 1) = 2022$, for $|m - n|$ to be as minimal as possible we need $m$ and $n$ as close as possible. The closest factors of $2022$ are $6$ and $337$, so answer is $337 - 6 = \boxed{331}$.
    \end{column}
    \begin{column}{0.5\textwidth}
      \begin{problem}
        \textbf{CP2.} (For fun) In the novel, “The Curious Incident of the Dog in the Nighttime,” a student in England taking his A-level college entrance exam in maths was given the following question: 
        
        Prove that a triangle with sides that can written in the form $n^2 +1$, $n^2 -1$ and $2n$ (where $n > 1$) is right-angled.
      \end{problem}
      Since $(n^2 + 1)^2 = n^4 + 2n^2 + 1 = n^4 - 2n^2 + 1 + 4n^2 = (n^2 - 1)^2 + (2n)^2$, and using \emph{Converse of Pythagoras Theorem} we got that triangle with sides  $n^2 +1$, $n^2 -1$ and $2n$ (where $n > 1$) is right-angled.
    \end{column}
  \end{columns}
\end{frame}

\begin{frame}{Challenge problems 3 - 4}
  \begin{columns}[T]
    \begin{column}{0.5\textwidth}
      \begin{problem}
        \textbf{CP3.} Let $m$ and $n$ be roots of the polynomial $x^2 - 60x + 899 = 0.$  What is $m^2 + n^2$? 
        
        (\emph{Hint}: think about how $m^2 + n^2$ can be rewritten in terms of the sum and product of the roots $m$ and $n$).
      \end{problem}

      $m^2 + n^2 = m^2 + 2mn + n^2 - 2mn = (m + n)^2 - 2mn = (60)^2 - 2\cdot 899 = 3600 - 1798 = \boxed{1802}$.
    \end{column}
    \begin{column}{0.5\textwidth}
      \begin{problem}
        \textbf{CP4.} Find all real values of $n$ such that $2^{2n} + 2^n + 1 = 73$.  
        
        (\emph{Hint}:  What substitution would turn this into a quadratic?)
      \end{problem}
      Let $y = 2^n$, then our equation turns into $y^2 + y + 1 = 73$ or $y^2 + y - 72 = (y - 8)(y + 9) = 0$. So $y = 8$ or $-9$. If $2^n = 8$, $n = \boxed{3}$, if $2^n = -9$ we don't have solutions since $2^n > 0$ for any real $n$.
    \end{column}
  \end{columns}
\end{frame}

% \begin{frame}{Title}
%   \begin{columns}[T]
%     \begin{column}{0.5\textwidth}
%     \end{column}
%     \begin{column}{0.5\textwidth}
%     \end{column}
%   \end{columns}
% \end{frame}

\end{document}