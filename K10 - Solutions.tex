\RequirePackage{luatex85}
\documentclass[9pt,aspectratio=169]{beamer}

\usepackage{luamplib}
  \mplibsetformat{metafun}
  \mplibtextextlabel{enable}
\everymplib{input mpcolornames; input repere; input macros; input featpost3Dplus2D; beginfig(1);}
\everyendmplib{endfig;}

\usetheme{graham}

\title{Kvantik problems,\\ October 2022}
% \subtitle[Graham Middle School]{Graham Middle School Math Olympiad Team}

\begin{document}
\maketitle

\begin{frame}{Problems 6-7}
  \begin{columns}[T]
    \begin{column}{0.5\textwidth}
      \vspace*{-6pt}
      \begin{problem}
        \textbf{P6.} There are thermometers in the living room, bedroom, and kitchen. The temperature in Celcius in the bedroom is always one degree higher than in the living room, and the temperature in the kitchen is one degree higher than in the bedroom. Peter wrote down thermometer readings in the morning, afternoon, and evening but made a typo in one number. As a result, he got the following numbers (not in order): $17$, $18$, $19$, $22$, $25$, $25$, $26$, $27$, $27$. Which number has a typo, and that it should be? Please explain your answer.
      \end{problem}
      Since three reading from one time period are forming $3$ consecutive numbers, for correct numbers we should have another number differ by $1$ or by $2$. So $\boxed{22}$ is the number with a typo, since there is no number with a difference of $1$ or $2$ with this number. Since from the rest numbers we can form $2$ triplets ($17-18-19$ and $25-26-27$), and the remaining numbers are $25$ and $27$, the right number should be $\boxed{26}$.
    \end{column}
    \begin{column}{0.5\textwidth}
      \vspace*{-6pt}
      \begin{problem}
        \begin{wrapfigure}{r}{20.00mm}
          \hspace{-0.6cm}
          \begin{mplibcode}
            u := 0.75cm;
            fill (0u, 0u)--(1u, 0u)--(2u, 1u)--(2u, 2u)--(1u, 3u)--(0u, 3u)--(-1u, 2u)--(-1u, 1u)--cycle withcolor yellow;
            fill (0u, 0u)--(1u, 0u)--(1u, 1u)--(0u, 1u)--cycle withcolor 0.4[blue,white];
            fill (1u, 1u)--(2u, 1u)--(2u, 2u)--(1u, 2u)--cycle withcolor 0.4[blue,white];
            fill (0u, 2u)--(1u, 2u)--(1u, 3u)--(0u, 3u)--cycle withcolor 0.4[blue,white];
            fill (-1u, 1u)--(0u, 1u)--(0u, 2u)--(-1u, 2u)--cycle withcolor 0.4[blue,white];
            draw (0u, 0u)--(1u, 0u)--(2u, 1u)--(2u, 2u)--(1u, 3u)--(0u, 3u)--(-1u, 2u)--(-1u, 1u)--cycle;
          \end{mplibcode}
          \hspace{0cm}
        \end{wrapfigure}
        \textbf{P7.} Masha stitched an eight-sided tablecloth from five squares and four isosceles right triangles (see picture). Is it possible to stitch the same tablecloth from one square and eight isosceles right triangles (not necessarily of the same size)?
      \end{problem}
      It can be cut like this.
      \begin{center}
        \leavevmode
        \begin{mplibcode}
          u := 1cm;
          fill (0u, 0u)--(1u, 0u)--(2u, 1u)--(2u, 2u)--(1u, 3u)--(0u, 3u)--(-1u, 2u)--(-1u, 1u)--cycle withcolor 0.5[yellow, white];
          fill (0u, 0u)--(1u, 0u)--(1u, 1u)--(0u, 1u)--cycle withcolor 0.7[blue,white];
          fill (1u, 1u)--(2u, 1u)--(2u, 2u)--(1u, 2u)--cycle withcolor 0.7[blue,white];
          fill (0u, 2u)--(1u, 2u)--(1u, 3u)--(0u, 3u)--cycle withcolor 0.7[blue,white];
          fill (-1u, 1u)--(0u, 1u)--(0u, 2u)--(-1u, 2u)--cycle withcolor 0.7[blue,white];
          draw (0u, 0u)--(1u, 0u)--(2u, 1u)--(2u, 2u)--(1u, 3u)--(0u, 3u)--(-1u, 2u)--(-1u, 1u)--cycle;
          draw (0u, 1u)--(1u, 0u)--(2u, 1u)--cycle penextrabold;
          draw (1u, 1u)--(2u, 2u)--(1u, 3u)--cycle penextrabold;
          draw (1u, 2u)--(0u, 3u)--(-1u, 2u)--cycle penextrabold;
          draw (0u, 0u)--(0u, 2u)--(-1u, 1u)--cycle penextrabold;
          draw (0u, 0u)--(1u, 0u) penextrabold; 
          draw (2u, 1u)--(2u, 2u) penextrabold; 
          draw (1u, 3u)--(0u, 3u) penextrabold; 
          draw (-1u, 2u)--(-1u, 1u) penextrabold; 
        \end{mplibcode}
      \end{center}
    \end{column}
  \end{columns}
\end{frame}

\begin{frame}{Problem 8}
  \begin{columns}[T]
    \begin{column}{0.5\textwidth}
      \begin{problem}
        \textbf{P8.} In the word $OBFUSCATION$, pupils randomly change letters for digits (the same letters changed to the same digits, different letters to different digits, and the first letter of the word can't be changed to digit $0$). Find the probability the resulting number is divisible by $3$. (The fraction of numbers divisible by $3$ from all possible numbers that can be obtained this way.)
      \end{problem}
      The number is divisible by $3$ when the sum of its digits is divisible by $3$.
      Since $BFUSCATION$ contains $10$ different letters, they should be $10$ different digits, so they sum is $0 + 1 + 2 + \ldots + 9 = 45$. And since $45$ is divisible by $3$, the reminder when $OBFUSCATION$ is divisible by $3$ is the same as the reminder when $O$ is divisible by $3$. $O$ may have $9$ values: $1$, $2$, $\ldots$, $9$, and only $3$ of them is divisible by $3$: $3$, $6$, $9$.

      So the probability that resulting number is divisible by $3$ is $3/9=\boxed{1/3}$.
    \end{column}
    \begin{column}{0.5\textwidth}
    \end{column}
  \end{columns}
\end{frame}

\begin{frame}{Problem 9}
  \begin{columns}[T]
    \begin{column}{0.5\textwidth}
      \begin{problem}
        \textbf{P9.} All faces of the triangular pyramid are equal equilateral triangles. On every face, all mid sides of the edges are connected, so each face is split into $4$ smaller triangles. Each of these $16$ triangles is painted in one of three colors --- red, blue, or green, so any two triangles with connected edges are painted in different colors (don't forget that triangles with the same edge may be from different faces). What is the largest possible number of red triangles?
      \end{problem}
      \begin{wrapfigure}{l}{20.00mm}
        \begin{mplibcode}
          u := 0.75cm;
          pair A, B, C;
          A := origin;
          B := (4u, 0u);
          C := 4u*dir(60);
          fill A--0.25[A, B]--0.25[A, C]--cycle withcolor 0.7white;
          fill B--0.25[B, A]--0.25[B, C]--cycle withcolor 0.7white;
          fill C--0.25[C, A]--0.25[C, B]--cycle withcolor 0.7white;
          fill 0.25[A, B]--0.5[A, 0.5[B, C]]--0.5[B, 0.5[A, C]]--0.75[A, B]--cycle withcolor 0.7white;
          fill 0.25[A, C]--0.5[A, 0.5[B, C]]--0.5[C, 0.5[A, B]]--0.75[A, C]--cycle withcolor 0.7white;
          fill 0.25[B, C]--0.5[B, 0.5[A, C]]--0.5[C, 0.5[A, B]]--0.75[B, C]--cycle withcolor 0.7white;
          draw 0.5[A, B]--0.5[A, C]--0.5[B, C]--cycle penbold;
          Draw A--B--C--cycle, 0.25[A, B]--0.25[A, C]--0.5[A, 0.5[B, C]]--cycle, 0.25[B, C]--0.25[B, A]--0.5[B, 0.5[A, C]]--cycle, 0.25[C, A]--0.25[C, B]--0.5[C, 0.5[A, B]]--cycle, 0.5[A, 0.5[B, C]]--0.5[B, 0.5[A, C]]--0.5[C, 0.5[A, B]]--cycle;
          drawdblarrow (1/8[A,B]){dir -45}..{dir 45}(7/8[A, B]);
          drawdblarrow (3/8[A,B]){dir -45}..{dir 45}(5/8[A, B]);
          drawdblarrow (1/8[A,C]){dir 105}..{dir 15}(7/8[A, C]);
          drawdblarrow (3/8[A,C]){dir 105}..{dir 15}(5/8[A, C]);
          drawdblarrow (1/8[B,C]){dir 75}..{dir 165}(7/8[B, C]);
          drawdblarrow (3/8[B,C]){dir 75}..{dir 165}(5/8[B, C]);
        \end{mplibcode}
      \end{wrapfigure}
      Let's look at the sweep of the pyramid. We will call triangles in the middle of the side a "middle triangle" (they are white in the diagram) and the triangles touching the corners of the side "corner triangles" (they are grey in the diagram). 
    \end{column}
    \begin{column}{0.5\textwidth}
      Three triangles in one corner should have different colors because they are connected to each other. So inside the grey area of the sweep can't be more than $4$ red triangles.
      Also, the remaining $4$ central triangles of each side can't be red if one of the triangles on this side is red, because it is connected to all $3$ corner triangles on this side.

      So, if we have $1$, $2$ or $3$ red corner triangles, the maximum number of middle red triangles is $3$. If we have $4$ red corner triangles, at least $2$ sides
      \begin{wrapfigure}{r}{20.00mm}
        \begin{mplibcode}
          u := 0.75cm;
          pair A, B, C;
          A := origin;
          B := (4u, 0u);
          C := 4u*dir(60);

          fill A--0.25[A, B]--0.25[A, C]--cycle withcolor red;
          fill 0.25[A, B]--0.25[A, C]--0.5[A, 0.5[B, C]]--cycle withcolor green;
          fill 0.25[A, B]--0.5[A, 0.5[B, C]]--0.5[A, B]--cycle withcolor red;
          fill 0.5[A, B]--0.5[A, 0.5[B, C]]--0.5[B, 0.5[A, C]]--cycle withcolor 0.25[blue,white];
          fill 0.5[A, B]--0.5[B, 0.5[A, C]]--0.75[A, B]--cycle withcolor green;
          fill 0.75[A, B]--0.25[B, C]--0.5[B, 0.5[A, C]]--cycle withcolor red;
          fill B--0.25[B, A]--0.25[B, C]--cycle withcolor green;

          fill 0.25[A, C]--0.5[A, 0.5[B, C]]--0.5[A, C]--cycle withcolor 0.25[blue,white];
          fill 0.5[A, C]--0.5[A, 0.5[B, C]]--0.5[C, 0.5[A, B]]--cycle withcolor green;
          fill 0.5[A, 0.5[B, C]]--0.5[B, 0.5[A, C]]--0.5[C, 0.5[A, B]]--cycle withcolor red;
          fill 0.5[B, C]--0.5[B, 0.5[A, C]]--0.5[C, 0.5[A, B]]--cycle withcolor green;
          fill 0.25[B, C]--0.5[B, 0.5[A, C]]--0.5[B, C]--cycle withcolor 0.25[blue,white];

          fill 0.25[C, A]--0.5[C, A]--0.5[C, 0.5[A, B]]--cycle withcolor red;
          fill 0.25[C, A]--0.25[C, B]--0.5[C, 0.5[A, B]]--cycle withcolor green;
          fill 0.5[C, B]--0.25[C, B]--0.5[C, 0.5[A, B]]--cycle withcolor red;

          fill C--0.25[C, A]--0.25[C, B]--cycle withcolor 0.25[blue,white];

          draw 0.5[A, B]--0.5[A, C]--0.5[B, C]--cycle penbold;
          Draw A--B--C--cycle, 0.25[A, B]--0.25[A, C]--0.5[A, 0.5[B, C]]--cycle, 0.25[B, C]--0.25[B, A]--0.5[B, 0.5[A, C]]--cycle, 0.25[C, A]--0.25[C, B]--0.5[C, 0.5[A, B]]--cycle, 0.5[A, 0.5[B, C]]--0.5[B, 0.5[A, C]]--0.5[C, 0.5[A, B]]--cycle;
          drawdblarrow (1/8[A,B]){dir -45}..{dir 45}(7/8[A, B]);
          drawdblarrow (3/8[A,B]){dir -45}..{dir 45}(5/8[A, B]);
          drawdblarrow (1/8[A,C]){dir 105}..{dir 15}(7/8[A, C]);
          drawdblarrow (3/8[A,C]){dir 105}..{dir 15}(5/8[A, C]);
          drawdblarrow (1/8[B,C]){dir 75}..{dir 165}(7/8[B, C]);
          drawdblarrow (3/8[B,C]){dir 75}..{dir 165}(5/8[B, C]);
        \end{mplibcode}
      \end{wrapfigure}
      should have red corner triangles and the maximum of middle red triangles is $2$. In both cases, the total maximum of red triangles is $\boxed{6}$.

      The example of the pyramid with $6$ red triangles is shown on the diagram.
    \end{column}
  \end{columns}
\end{frame}

\begin{frame}{Problem 10}
  \begin{columns}[T]
    \begin{column}{0.5\textwidth}
      \begin{problem}
        \textbf{P10.} Is there a polygon that can be cut with one straight line into triangles of areas $1$, $2$, and $3$, and with another straight line into triangles of areas $2$, $2$, and $2$?
      \end{problem}
      \begin{center}
        \leavevmode
        \begin{mplibcode}
          u = 0.7cm;
          draw (0u, 0u)--(4u, 0u)--(4u, 4u)--(2u, 1u)--(0u, 6u)--cycle penextrabold;
          draw ddline((0u, 0u),(4u, 2u))(0.25, 0.25) withcolor red pensemibold;
          draw ddline((0u, 2u),(4u, 0u))(0.25, 0.25) withcolor blue pensemibold;
          label.("1", incenter((0u, 0u), (2u, 1u), (4u, 0u)));
          label.("1", incenter((0u, 0u), (2u, 1u), (0u, 2u)));
          label.("1", incenter((4u, 0u), (2u, 1u), (4u, 2u)));
          label.("1", incenter((4u, 2u), (2u, 1u), (4u, 4u)));
          label.("2", incenter((0u, 2u), (2u, 1u), (0u, 6u)));
          label.bot("$2\sqrt{2}$", (2u, 0u));
          label.lft("$\sqrt{2}$", (0u, 1u));
          label.rt("$\sqrt{2}$", (4u, 1u));
          label.lft("$2\sqrt{2}$", (0u, 4u));
          label.rt("$\sqrt{2}$", (4u, 3u));
        \end{mplibcode}
      \end{center}
    \end{column}
    \begin{column}{0.5\textwidth}
      One example of a such polygon is shown on the left. The areas of smaller triangles are shown in black, and the length of the sides is shown in blue. The red line divides the polygon into triangles of sides $3$, $2$ and $1$, and the blue line divides the polygon into $3$ triangles of side $2$.

      The height of the triangle with base on the bottom of the polygon is $\dfrac{\sqrt{2}}{2}$, so the area of the triangle is \[\dfrac{1}{2} bh = \dfrac{1}{2} \cdot 2\sqrt{2} \cdot \dfrac{\sqrt{2}}{2} = 1.\] 
      For triangles with the base on the vertical sides of the polygon, the height is $\sqrt{2}$, so their areas are $\dfrac{1}{2} \cdot \sqrt{2} \cdot \sqrt{2} = 1$ or $\dfrac{1}{2} \cdot 2\sqrt{2} \cdot \sqrt{2} = 2$ as shown.
    \end{column}
  \end{columns}
\end{frame}

% \begin{frame}{Title}
%   \begin{columns}[T]
%     \begin{column}{0.5\textwidth}
%     \end{column}
%     \begin{column}{0.5\textwidth}
%     \end{column}
%   \end{columns}
% \end{frame}


% \begin{frame}{Title}
%   \begin{columns}[T]
%     \begin{column}{0.5\textwidth}
%     \end{column}
%     \begin{column}{0.5\textwidth}
%     \end{column}
%   \end{columns}
% \end{frame}

\end{document}
