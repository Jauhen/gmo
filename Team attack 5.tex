\documentclass[11pt]{article}

\usepackage{amsmath}
\usepackage{luamplib}
\mplibsetformat{metafun}
\mplibtextextlabel{enable}
\everymplib{input mpcolornames; input repere; input macros; beginfig(1);}
\everyendmplib{endfig;}

\usepackage[landscape,twocolumn, margin=0.75in, top=0.5in]{geometry}
\setlength{\columnsep}{1.5in}
\usepackage{enumitem}

\begin{document}
  \pagestyle{empty}
  \section*{Team attack 11/09/2022}
  \begin{enumerate}[leftmargin=3mm]
    \item $\dfrac{9}{37}$ is changed to a decimal. What digit lies in the 2022\textsuperscript{th} place to the right of the decimal point? %Set 9, 1D
    \item Emily has 21 dimes. She placed then in three piles, with an odd number of dimes in each pile. In how many different ways can she accomplish this? [\emph{Consider piles of 1, 1, 19 dimes, for example, to be equivalent to piles 1, 19, and 1 dimes.}] % Set 9, 2B
    \item Suppose $\dfrac{2}{N}$, $\dfrac{3}{N}$, and $\dfrac{5}{N}$ are three fractions in lowest terms. Find a sum of all the possible composite whole number values for $N$ between $20$ and $80$? % Set 9, 2D
    \item The Mathematical Olympiad began in the prime year 1979. Find the product of the fractions below in a simplest form:
    \[ 
      \left(1 - \frac{1}{1980}\right) \times \left(1 - \frac{1}{1981}\right) \times \ldots \times \left(1 - \frac{1}{2022}\right). 
    \] % Set 9, 2E
    \item In this street map, all traffic enters at A and exits at either B or C. All traffic flows either south or east. At each intersection where there is a choice of direction, 70\% of the traffic goes east and 30\% goes south. What percent of the traffic exists at C? % Set 9, 3D
    \item Rectangle $ABCD$ is partitioned into five squares as shown. The length, in centimeters, of $\overline{AM}$ is a whole number. The area of rectangle $ABCD$ is greater than 100 sq cm. Find the smallest possible area of rectangle $ABCD$, in sq cm. % Set 11, 2D
    \item Two semicircles are inscribed in a square with side $8$ meters as shown. Approximate the area of the shaded region to the nearest tenth of a square meter. Use the approximation $3.14$ for $\pi$. % Set 9, 5D.
  \end{enumerate}
  \begin{tabular}{ccccc}
    \begin{mplibcode}
      u := 0.6cm;
        drawarrow (-1u, 2u)--(-.1u, 2u) penbold;
        drawarrow (0u, 2u)--(1.9u, 2u) penbold;
        drawarrow (2u, 2u)--(3u, 2u) penbold;
        drawarrow (0u, 1u)--(1.9u, 1u) penbold;
        drawarrow (0u, 0u)--(1.9u, 0u) penbold;
        drawarrow (2u, 0u)--(3u, 0u) penbold;
        drawarrow (0u, 2u)--(0u, 1.1u) penbold;
        drawarrow (2u, 2u)--(2u, 1.1u) penbold;
        drawarrow (0u, 1u)--(0u, 0.1u) penbold;
        drawarrow (2u, 1u)--(2u, 0.1u) penbold;
        label.lft("$A$", (-1u, 2u));
        label.rt("$B$", (3u, 2u));
        label.rt("$C$", (3u, 0u));
    \end{mplibcode}&\quad&
    \begin{mplibcode}
      u := 0.25cm;
      Draw (0u, 0u)--(7u, 0u)--(7u, 4u)--(0u, 4u)--cycle, (3u, 0u)--(3u, 4u), (0u, 3u)--(3u, 3u), (1u, 3u)--(1u, 4u), (2u, 3u)--(2u, 4u);
      label.ulft("$A$", (0u, 4u));
      label.lft("$M$", (0u, 3u));
      label.llft("$D$", (0u, 0u));
      label.urt("$B$", (7u, 4u));
      label.lrt("$C$", (7u, 0u));
    \end{mplibcode}&\quad&
    \begin{mplibcode}
      u := 0.7cm;
      fill (0u, 0u)--(2u, 0u)--(2u, 2u)--(0u, 2u)--cycle withcolor 0.7white;
      fill subpath (2, 4) of circle((1u, 2u), 1u)--cycle withcolor white;
      fill subpath (3, 5) of circle((0u, 1u), 1u)--cycle withcolor white;
      draw (0u, 0u)--(2u, 0u)--(2u, 2u)--(0u, 2u)--cycle;
      draw subpath (2, 4) of circle((1u, 2u), 1u);
      draw subpath (3, 5) of circle((0u, 1u), 1u);
    \end{mplibcode}\\
    Problem 5&\quad&Problem 6&\quad&Problem 7        
  \end{tabular}

  \newpage
  \section*{Team attack 11/09/2022}
  \begin{enumerate}[leftmargin=3mm]
    \item $\dfrac{9}{37}$ is changed to a decimal. What digit lies in the 2022\textsuperscript{th} place to the right of the decimal point? %Set 9, 1D
    \item Emily has 21 dimes. She placed then in three piles, with an odd number of dimes in each pile. In how many different ways can she accomplish this? [\emph{Consider piles of 1, 1, 19 dimes, for example, to be equivalent to piles 1, 19, and 1 dimes.}] % Set 9, 2B
    \item Suppose $\dfrac{2}{N}$, $\dfrac{3}{N}$, and $\dfrac{5}{N}$ are three fractions in lowest terms. Find a sum of all the possible composite whole number values for $N$ between $20$ and $80$? % Set 9, 2D
    \item The Mathematical Olympiad began in the prime year 1979. Find the product of the fractions below in a simplest form:
    \[ 
      \left(1 - \frac{1}{1980}\right) \times \left(1 - \frac{1}{1981}\right) \times \ldots \times \left(1 - \frac{1}{2022}\right). 
    \] % Set 9, 2E
    \item In this street map, all traffic enters at A and exits at either B or C. All traffic flows either south or east. At each intersection where there is a choice of direction, 70\% of the traffic goes east and 30\% goes south. What percent of the traffic exists at C? % Set 9, 3D
    \item Rectangle $ABCD$ is partitioned into five squares as shown. The length, in centimeters, of $\overline{AM}$ is a whole number. The area of rectangle $ABCD$ is greater than 100 sq cm. Find the smallest possible area of rectangle $ABCD$, in sq cm. % Set 11, 2D
    \item Two semicircles are inscribed in a square with side $8$ meters as shown. Approximate the area of the shaded region to the nearest tenth of a square meter. Use the approximation $3.14$ for $\pi$. % Set 9, 5D.
  \end{enumerate}
  \begin{tabular}{ccccc}
    \begin{mplibcode}
      u := 0.6cm;
        drawarrow (-1u, 2u)--(-.1u, 2u) penbold;
        drawarrow (0u, 2u)--(1.9u, 2u) penbold;
        drawarrow (2u, 2u)--(3u, 2u) penbold;
        drawarrow (0u, 1u)--(1.9u, 1u) penbold;
        drawarrow (0u, 0u)--(1.9u, 0u) penbold;
        drawarrow (2u, 0u)--(3u, 0u) penbold;
        drawarrow (0u, 2u)--(0u, 1.1u) penbold;
        drawarrow (2u, 2u)--(2u, 1.1u) penbold;
        drawarrow (0u, 1u)--(0u, 0.1u) penbold;
        drawarrow (2u, 1u)--(2u, 0.1u) penbold;
        label.lft("$A$", (-1u, 2u));
        label.rt("$B$", (3u, 2u));
        label.rt("$C$", (3u, 0u));
    \end{mplibcode}&\quad&
    \begin{mplibcode}
      u := 0.25cm;
      Draw (0u, 0u)--(7u, 0u)--(7u, 4u)--(0u, 4u)--cycle, (3u, 0u)--(3u, 4u), (0u, 3u)--(3u, 3u), (1u, 3u)--(1u, 4u), (2u, 3u)--(2u, 4u);
      label.ulft("$A$", (0u, 4u));
      label.lft("$M$", (0u, 3u));
      label.llft("$D$", (0u, 0u));
      label.urt("$B$", (7u, 4u));
      label.lrt("$C$", (7u, 0u));
    \end{mplibcode}&\quad&
    \begin{mplibcode}
      u := 0.7cm;
      fill (0u, 0u)--(2u, 0u)--(2u, 2u)--(0u, 2u)--cycle withcolor 0.7white;
      fill subpath (2, 4) of circle((1u, 2u), 1u)--cycle withcolor white;
      fill subpath (3, 5) of circle((0u, 1u), 1u)--cycle withcolor white;
      draw (0u, 0u)--(2u, 0u)--(2u, 2u)--(0u, 2u)--cycle;
      draw subpath (2, 4) of circle((1u, 2u), 1u);
      draw subpath (3, 5) of circle((0u, 1u), 1u);
    \end{mplibcode}\\
    Problem 5&\quad&Problem 6&\quad&Problem 7             
  \end{tabular}

\end{document}